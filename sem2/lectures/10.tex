
\subsection{Касательное пространство к многообразию}

\begin{definition} (Касательный вектор к многообразию)

Пусть $\M$ простое $k$ мерное $r$ гладкое многообразие в $\R^{m}$. Будем говорить, что вектор $\tau$ является касательным к $\M$ в точке $p$, Если $\exists \gamma \in C^{1} ((a;b), \M)$ такое, что при каком-то $t^{0} \in (a; b)$ и $\gamma(t^{0}) = p$
выполнено $\gamma' (t^0) = \tau \in \R^m$ 


\end{definition}
    
    
\begin{definition} (Касательное пространство к многообразию)
    Пусть $\M$ простое $k$ мерное $r$ гладкое многообразие в $\R^{m}$. Касательным пространством к многообразию $\M$ в точке $p$ называют множество всех касательных векторов в точке $p$. И обозначается $T_{p} \M$
    
  
\end{definition}  
    
    

Выберем точку $p \in \M$ на многообразии. По определению существует ее окрестность $U(p) \in \R^{m}$, такая что $U(p) \cap \M$  простое $k$ мерное $r$ гладкое многообразие. По определению у него существует параметризация $\exists \Phi \in C^{r} (\G, \M)$ $\Phi$~---~гомеоморфизм $\G$ на $U(p) \cap \M$ По определению $\rank \D \Phi = k, \forall a \in \G$


\textbf{Картиночка будет в следующий раз}

Получается у нас есть какая-то область $\G$ которая под действием гомеоморфизма отображается в наше простое многообразие. В этой области существует точка, образ которой это точка $p$. То есть 	$\exists a = (a_1, \dots , a_k) \in \G: \Phi(a) = p$.
Зафиксируем все координаты $a$, кроме $a_j$, его будем изменять. Мы рассматриваем кривую $\gamma^{j}(t) = \Phi (a_1, \dots, a_{j-1}, t, a_{j+1}, a_k)$ она называется j-ая координатная линия. Касательный вектор к этой кривой называется j канонический касательным вектором к многообразию $\M$ в точке $p$

    
\[
\tau_{j} := \frac{d\gamma^{j}}{d t} (a_{j}) =  
        \begin{pmatrix}
            \frac{\partial \Phi_1}{\partial x_j} (a_1, \dots, a_{k}) \\
            \vdots \\
            
            \frac{\partial \Phi_m}{\partial x_j} (a_1, \dots, a_{k})
        \end{pmatrix} \text{j столбец матрицы Якоби отображения } \Phi 
\]




Получается геометрическая интерпретация столбцов матрицы Якоби — это канонические касательные вектора. С другой стороны, если посмотреть на матрицу Якоби, как на матрицу дифференциала. А дифференциал это линейное отображение, которое приближает с точностью до о малого. Любому линейному отображению соответствует матрица и наоборот. Получается, что j столбец — это образ j базисного вектора. То ест $\tau_{j} = d_{a} \Phi (e_{j})$. Ранг матрицы дифференциала равен $k$, $e_j$ - это $k$ линейно независимых векторов. Значит $\tau_1, \dots, \tau_k$ — линейно независимые вектора

\begin{theorem} (Корректность определения касательного пространства)

Пусть $\M$ — это $k$-мерное $r$-гладкое многообразие $\M \in \R^{m}, 1 \leq k \leq m, k \in [1, +\infty], p \in \M$. Тогда $T_{p} \M$ является линейным подпространством в $\R^{m}$ размерности $\R^k$ 
\end{theorem}

\begin{proof}
По определению многообразия $\exists U(p) \in \R^{m}$, такое что $U(p) \cap \M$ — простое $k$ мерное $r$ гладкое многообразие

По определению у него существует локальная параметризация $\exists \Phi \in C^{r} (\G, \M), \Phi(\G) = U(p) \cap \M $ $\Phi$ — гомеоморфизм $\G$ на $U(p) \cap \M$ По определению $\rank \D \Phi = k, \forall a \in \G$

Пусть есть два касательных вектора $\tau, \tau_{*} \in T_p \M$. Тогда по определению для каждого из них существуют кривые $\gamma$ и $\gamma_{*}$. Выберем их так, чтобы они были определены на одном и том же интервале, содержащем точку 0(Мы всегда можем этого добиться допустимой заменой и сдвигом аргументов)

$
\left\{ \begin{aligned} 
  & \gamma, \gamma_* \in C^{1} ( (-\delta, \delta) ,\ \M )  \\
  \gamma(0) &= \gamma_{*}(0) = p \\
  \gamma'(0) &= \tau \\
  \gamma_{*}'(0) &= \tau_{*} 
\end{aligned} \right.
$

Мы хотим доказать, что $T_p \M$ обладает структурой линейного пространства.

Нулевой вектор, очевидно принадлежит пространству, достаточно взять стоячую кривую.

Рассмотрим линейную комбинацию $\alpha \tau + \beta \tau_{*}$. Мы хотим доказать, что она принадлежит этому пространству 

Мы не можем легко предъявить кривую касательный вектор к которой был бы этой линейной комбинацией, так как мы не можем складывать кривые на многообразии. Поэтому поступим хитрее.
\[
\text{Рассмотрим кривые: }
\begin{cases}
    \widetilde{\gamma} = \Phi^{-1} \circ \gamma  \\
    \widetilde{\gamma_{*}}  = \Phi^{-1} \circ \gamma_{*} \\
     \widetilde{\gamma}(0) = \widetilde{\gamma_{*}}(0) = \Phi ^{-1}(p) = a
\end{cases}
\text{(Англ. Pull back)}
\]


По теореме о продолжении до диффеоморфизма, можно без ограничения общности считать, что $\Phi$ не просто отображает из $\G$ в многообразие, но является $r$-гладкий диффеоморфизм некого открытого множества $\Omega \in \R^{m}$, $\Phi(\Omega) \in \R^m$ - тоже открытое

Получаем, что у нас есть гладкие кривые $
\left\{ \begin{aligned} 
  \widetilde{\gamma} \in C^{1} ((-\delta, \delta), \G) \\
  \widetilde{\gamma_{*}} \in C^{1} ((-\delta, \delta), \G)
\end{aligned} \right.
$



 (Кривые $\gamma, \gamma_{*}$ лежат в $U(p) \cap \M$, но $\Phi$ действовало из $\G$, но так мы действуем на них обратным отображением, мы попадаем в $\G$)

По теореме о дифференцировании композиции: $
\left\{ \begin{aligned} 
  \widetilde{\tau} &:= \widetilde{\gamma} (0)' = \D \Phi^{-1} (p) \circ \gamma' (0) \\
  \widetilde{\tau_{*}} &:= \widetilde{\gamma_{*}} (0)' = \D \Phi^{-1} (p) \circ \gamma_{*}' (0)
\end{aligned} \right.
$


Мы перетащили наши кривые в область $\G \in \R^k$, в которой можно складывать вектора(можно также поточечно сложить и кривые, полученная кривая, хотя бы ее часть, содержащая точку a, также будет в $\G$). А вот на многообразии нельзя было складывать вектора, так как их сумма могла оказаться вне многообразия.


По условию $\widetilde{\gamma}, \widetilde{\gamma_{*}} \in \G \Longrightarrow$ так как $\G$ - открытое множество $\exists \widetilde{\delta} \in (0, \delta)$, такой, что $\alpha \widetilde{\gamma}(t) + \beta \widetilde{\gamma_{*}}(t) \in \G \ \forall t \in (-\widetilde{\delta}, \widetilde{\delta})$. Мы получили линейную комбинацию кривых, и теперь мы подействуем обратно отображением и "толкнем"(Англ. Push forward) на многообразие.

Рассмотрим кривую $$\hat{\gamma} := \Phi (\alpha \widetilde{\gamma} + \beta \widetilde{\gamma_{*}}) \in C^1 (((-\widetilde{\delta}, \widetilde{\delta})), \ \M ).$$
Покажем, что это искомая кривая, то есть касательный вектор в точке $p$ является искомой линейной комбинацией (Здесь $\Phi$ — продолженная до диффеоморфизма матрица и $p = \Phi (a)$)


\begin{flushleft}
    $ \hat{\gamma}' = \D \Phi (a) \cdot (\alpha \widetilde{\gamma}'(0) + \beta \widetilde{\gamma_{*}}'(0)) =
\D \Phi (a) \cdot (\alpha \D \Phi^{-1} (p) \circ \gamma' (0) + \beta \D \Phi^{-1} (p) \circ \gamma_{*}' (0)) = $
\end{flushleft}

\begin{flushright}
$ = \D \Phi (a) \cdot \D \Phi^{-1} (\Phi (a)) [\alpha \gamma'(0) + \beta \gamma_{*}'(0)] =  E \cdot [\alpha \gamma'(0) + \beta \gamma_{*}'(0)] = \alpha \tau + \beta \tau_{*}$
\end{flushright}
То есть мы построили искомую кривую. В итоге $\forall \alpha, \ \beta \in \R, \ \forall \tau, \ \tau_{*} \in T_{p} \M \hookrightarrow \alpha \tau + \beta \tau_{*} \in T_{p} \M \Longrightarrow$ Следовательно касательное пространство действительно является линейным подпространством в $\R^{m}$. 

 Определим его размерность. Заметим, что $\dim T_p \M \geq k$, так как мы показали что есть канонические касательные вектора в количестве $k$ штук, которые линейно независимы. Докажем что размерность $T_p \M \leq k$. 
$\forall t \in \R^k \hookrightarrow t = t_1 e_1 + \dots + t_k e_k$

$
\left\{ \begin{aligned} 
\widetilde{\gamma} &= \Phi^{-1} \circ \gamma \\
\widetilde{\gamma} (0)' &= \D \Phi^{-1} (p) \circ \gamma' (0)
\end{aligned} \right.
$




$$ d_a \Phi (t) = t_1 \cdot d_a \Phi (e_1) + \dots + t_k \cdot d_a \Phi (e_k) = t_1 \tau_1 + \dots + t_k \tau_k $$

$$\widetilde{\gamma}' = (\D \Phi ^{-1}) (p) (t_1 \tau_1 + \dots + t_k \tau_k)$$

$$d_a \Phi (\widetilde{\gamma}') = \D \Phi (a) \widetilde{\gamma}'(0) = t_1 \tau_1 + \dots + t_k \tau_k) = \tau$$

Получается  $T_p \M \in d_a \Phi (\R^k)$ из построения кривых $\widetilde{\gamma}$ следует, что касательное пространство содержится в образе дифференциала отображения $\Phi$ (Для любого вектора из $T_p \M$ мы находи кривую pull back, касательный к которой лежит в $\R^k$ что и означает что размерность не более $k$). Получаем, что $\dim T_p \M = k$

\end{proof}


\begin{definition} (Множество уровня)
Пусть задана скалярная функция из какого-то открытого и непустого множества $F \in C^r (\Omega \in \R^m, \R)$. Множеством уровня назовем
$$M := \{x: F(x) = C \in \R\}$$
\end{definition}

\begin{explanation} (Связь множеств уровня и многообразий)

Пусть у нас дана какая-то гладкая кривая $\gamma (-\delta, \ \delta) \mapsto M$, $\gamma'(0) \neq 0$

Рассмотрим функцию  $
\left\{ \begin{aligned} 
  \psi(t) &= F(\gamma(t)) = C \\
  \gamma(0) &= p \in M
\end{aligned} \right.
$ $\Longrightarrow \psi' = 0$

Предположим, что $\grad F(p) \neq 0$

$\psi'(0) = \langle \grad F(p), \gamma'(0) \rangle = 0 \Longrightarrow \gamma'(0) \perp \grad F(p)$

Следовательно, множество уровня — это гладкое многообразие. Если градиент всюду не вырождается, получается что касательное пространство - это пространство размерность $m-1$ перпендикулярное градиенту.
\end{explanation}

Если вспомнить эквивалентные определения многообразия:
 $\exists V(p) \subset \R^m$~---~окрестность точки $p$, такая что $\mathcal{V}(p) \cap \M \Longleftrightarrow$

 Существует открытая окрестность $U(p) \in \R^m$ и такие определенные в ней функции $F_{1}, \dots , F_{m-k} \in C^{r}(U(p), R)$, такие, что $\forall x \in U(p)$ выполнено:
    $$x \in \M \Longleftrightarrow F_{1}(x) = \dots = F_{m-k} = 0$$
    А также векторы $\{grad F_1\Bigr|_{U(p) \cap \M}(p),\dots, \ grad F_{m-k}\Bigr|_{U(p) \cap \M}(p)\}$ линейно не зависимы

    Следовательно, локально простое гладкое многообразие является пересечением множеств уровня гладких функций.

    Работая с этим определением многообразия можно эквивалентно сказать, что $\M$ — это простое гладкое многообразие, если для каждой точки найдется окрестность и набор функций, таких что пересечение этой окрестности с многообразием это нули этих функций. Рассмотрим множество определённое таким образом:
    $$\M  = {x \in U(p): F_{1}(x) = \dots = F_{m-k} = 0}$$
    Рассмотрим аналитически касательное пространство к нему. Из пояснения следует что, если градиент не вырожден, то все касательные векторы к множеству уровня каждой функции должны быть перпендикулярны ее градиенту. Пересекая касательные пространства к множеству уровня каждой функции, мы получим касательное пространство в данной точке

    $$T_p \M = \{h \in \R^m: \left\{ 
    \begin{aligned} 
  \frac{\partial F_1}{\partial x_1}(p)\cdot h_1 + \dots +\frac{\partial F_1}{\partial x_m}(p)\cdot h_1 &= 0 \\
  \dots \\
  \frac{\partial F_{m-k}}{\partial x_1}(p)\cdot h_1 + \dots +\frac{\partial F_{m-k}}{\partial x_m}(p)\cdot h_1 &= 0 
\end{aligned} \right.
\}$$
Данная система линейна, ранг этой матрицы $k$ размерность касательного пространства $k$, как и было доказано

\subsection{Условный экстремум}

Будем работать с частным случаем гладкого многообразия. Будем считать что у нас есть открытое и непустое множество $\Omega  \subset \R^m$, $\M \subset \Omega$ $k$-мерное гладкое многообразие, задаваемое набором функций:
$ \phi_1, \dots \phi_k \in C^1(\Omega)$ (Ограничения или уравнения связи)
То есть, у нас дано какое-то множество(поверхность) $\M$, которое лежит в какой-то большей области, в которой заданы гладкие функции, Множество $\M$ определяется как множество нулей этих функций
$$\M := \{ p \in \Omega: \phi_1(p) = \dots = \phi_k(p) = 0 \}$$

Градиенты этих функций линейно не зависимы для любой точки на множестве $\M$, далее будем работать в таком предположении

\begin{definition} (Условный экстремум)

Условный экстремум — это экстремум функции $f: \M \mapsto \R$  на $k$ мерном гладком многообразии (Но мы будем считать что она задана на $\Omega$). Он называется условным, так как у нас появляются условия, заданные уравнениями $\phi_1, \dots, \phi_k$
    
\end{definition}

\begin{definition}
    Точку $p \in \M$ будем называть условным локальным max/min, если $\exists U(p) \in \R^m$ если $$\forall q \in U(p) \cap \M \hookrightarrow f(q) \leq f(p)$$ Отличие от обычного экстремума в том, что теперь мы рассматриваем точки которые лежат на пересечении некоторой обрасти с многообразием, порожденным условиями связи
\end{definition}

\begin{definition} (Функция Лагранжа)
Все что было описано в данном параграфе нам дано, $1 \leq k < m$ функцией Лагранжа называют:
$$\mathcal{L} (x, \lambda) := f(x) - \lambda_1 \phi_1(x) - \dots - \lambda_k \phi_k(x) = f - \lambda ^{T} \phi$$

В записи можно использовать как плюсы, так и минусы. ''$x$'' - это точка в $\Omega$, а $\lambda$ - это параметр в $\R^k$
\end{definition}


\begin{theorem}
    
\end{theorem}