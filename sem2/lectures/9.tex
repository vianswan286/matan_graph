\subsection{Экстремумы}

Экстремумы бывают \begin{itemize}
    \item \underline{Безусловные} (на множество определения функции не накладывается дополнительных ограничений)
    \item \underline{Условные} (на множество определения накладываются условия связи или уравнния связи, то есть фактически экстремум ищется на $k$-мерном многообразии)
\end{itemize}

\begin{definition}
    Пусть $E \subset \R ^ m$~---~непустое множество, $x^0 \in E$, $f$: $E \mapsto \R$. \\ Будем говорить, что $x^0$~---~\textit{точка локального экстремума} функции $f$ на множестве $E$, если $\exists \ \delta_0 > 0$, такая что $\forall \ x \in U_{\delta_0}(x^0) \cap E \hookrightarrow$ \\
    $ f(x) \le f(x^0)$ (точка локального минимума $f$ на $E$); \\
    $ f(x) \ge f(x^0)$ (точка локального максимума $f$ на $E$); \\
    в случае строгих неравенств говорят о точках \textit{строгого локального минимума/максимума}.
\end{definition}

Далее $E$~---~непустое открытое множество.

\subsection{Необходимое условие экстремума}

\begin{theorem}
    Пусть $E \neq \varnothing$, $E \in \R^m$~---~открытое. Пусть $f: \ E \mapsto \R$. \\ Пусть $\forall i \in \left\{ 1, \dots, m\right\} \ \exists \dfrac{\partial f}{\partial x_i} (x^0) \in \R$. Тогда если $x^0$~---~точка локального экстремума функции $f$ на $E$, то $$\dfrac{\partial f}{\partial x_i}(x^0) = 0 \ \forall i \in \left\{1, \ldots, m \right\}.$$
\end{theorem}

\begin{proof}
    Так как $E$~---~открыто и непусто, $\exists \delta > 0$, такое что $B_\delta(x^0) \subset E$. С другой стороны $x^0 \in E$~---~локальный экстремум $\Longrightarrow \ \exists \delta^0 > 0$, такое что 
    $$(\star)
        \left[
        \begin{array}{l}
        f(x) \le f(x^0) \ \forall x \in B_{\delta^0}(x^0) \cap E, \\
        f(x) \ge f(x^0) \ \forall x \in B_{\delta^0}(x^0) \cap E.
        \end{array}
        \right.
    $$
    Положим $\underline{\delta} = \min \left\{ \delta, \delta^0 \right\} \ \Longrightarrow \ B_{\underline{\delta}} \subset E$ и при этом выполняется система $(\star)$. \\
    Зафиксируем $i \in \left\{1, \dots, m \right\} $ и рассмотрим функцию 
    $$g_i(t) = f\left(x_1^0, \dots x_{i - 1}^0, t, x_{i + 1}^0, \dots, x_m^0 \right).$$
    У функции $g_i$ в точке $x_i^0$ есть локальный экстремум. \\
    Но $\exists \dfrac{\partial f}{\partial x_i}(x^0)\ \Longrightarrow \ \exists \dfrac{\partial g_i}{\partial t} (x_i^0)$.
    В силу необходимого условия экстремума для функции одной переменной $\dfrac{d g_i}{d t} (x^0) = 0 \ \Longrightarrow \ \dfrac{\partial f}{\partial x_i} (x^0) = 0$, \\
    но $i$ был выбран произвольно $\Longrightarrow \ \dfrac{\partial f}{\partial x_1}(x^0) = \dots = \dfrac{\partial f}{\partial x_m}(x^0)$.
\end{proof}

\subsection{Достаточное условие экстремума}
\begin{definition}
    \textit{Квадратичной формой} в $R^m$ назовём функцию $$K(h) := \sum\limits_{i=1}^m \sum\limits_{j=1}^m a_{ij} h_i h_j, \ a_{ij} = a_{ji}.$$
\end{definition}

$\begin{pmatrix}
  a_{11} & \dots & a_{1m} \\
 \vdots & & \vdots \\
  a_{m1} & \dots & a_{mm}
\end{pmatrix} $
~---~ матрица квадратичной формы (симметрична)
\begin{definition}
    $K(h)$ называется \textit{положительно определённой}, если $\forall h \neq 0 \hookrightarrow K(h) > 0$, \\
    отрицательно определённой, если $\forall h \neq 0 \hookrightarrow\ K(h) < 0$, \\
    знаконеопределённой, если $\exists h_1 \neq 0 K(h_1) > 0$ и $\exists h_2 \neq 0 K(h_1) < 0$.
\end{definition}

В случае, когда $f$ дважды дифференцируема в точке $x^0 \in E$ и все частные производные второго порядка непрерывны в точке $x^0$, то 
$$ d^2f(x^0) := \sum\limits_{i=1}^m \sum\limits_{j=1}^m \dfrac{\partial^2 f}{\partial x_i \partial x_j} (x^0) dx_i dx_j, $$ где $dx_i = x_i - x_i^0, \ dx_j = x_j - x_j^0.$ Так как все частные производные второго порядка непрерывны в точке $x_0$, получается:
$$ \dfrac{\partial^2 f}{\partial x_i \partial x_j} (x^0) = \dfrac{\partial^2 f}{\partial x_j \partial x_i} (x^0) $$

\begin{definition}
    \textit{Матрица Гессе (гессиан)}
    $$H_f(x^0) := 
    \begin{pmatrix}
        \dfrac{\partial^2 f}{\partial x_1 \partial x_1} (x^0) & \dots & \dfrac{\partial^2 f}{\partial x_1 \partial x_m} (x^0) \\
        \vdots & & \vdots \\
        \dfrac{\partial^2 f}{\partial x_m \partial x_j} (x^0) & \dots & \dfrac{\partial^2 f}{\partial x_m \partial x_m} (x^0)
    \end{pmatrix}$$
\end{definition}

\begin{theorem}
   (Критерий Сильвестра) 
   Квадратичная форма положительно определена $\Longleftrightarrow$ все главные миноры матрицы квадратичной формы $A$ положительны. \\
   Квадратичная форма отрицательно определена $\Longleftrightarrow$ знаки главных миноров чередуются, начиная с отрицательного. (Доказательство в курсе линейной алгебры)
\end{theorem}

\begin{theorem}
    Пусть $f \in C^2\left( B_{\delta^0}(x^0)\right)$, $\delta > 0$. Пусть в точке $x^0$ все частные производные первого порядка обращаются в ноль (то есть $x^0$~---~\underline{стационарная} точка функции $f$). \\
    Тогда \begin{enumerate}
        \item  Если квадратичная форма второго дифференциала $d^2_{x^0} f(dx)$ в точке $x^0$ положительно определена, то $x^0$~---~локальный минимум функции $f$.
        \item Если $d^2_{x^0} f(dx)$ отрицательно определена, то $x^0$~---~точка локального максимума функции $f$.
        \item Если $d^2_{x^0} f(dx)$ знаконеопределена, то экстремума нет.
        \item В остальных случаях ничего сказать нельзя.
    \end{enumerate}
\end{theorem}

\begin{example} (для пункта 4) \\
    $f(x_1, x_2) = x_1^4$ \\
    точка $(0, 0)$~---~локальный минимум и $H[f] (0, 0) \equiv 0$. \\
    $\widetilde{f}(x_1, x_2) = x_1^3$ \\
    $H[\widetilde{f}] \equiv 0$, но точка $(0, 0)$ не является точкой локального экстремума.
\end{example}

\begin{proof}
    Так как $f \in C^2\left(B_{\delta}(x^0)\right) \ \Longrightarrow$ справедлива формула Тейлора с остаточным членом в форме Пеано,
    $$ f(x) = f(x^0) + d_{x^0} f(dx) + \dfrac{1}{2} d^2_{x^0} f(dx) + o(\| dx \| ^2), \ x \to x^0 .$$
    Так как $x^0$~---~стационарная точка функции, $f$ 
    $$ f(x) = f(x^0) + \dfrac{ d^2_{x^0} f(dx)}{2} + o(\| x - x^0\| ^2), \ x \to x^0 $$

\begin{enumerate}
    \item Пусть $d^2_{x^0} f(dx)$~---~положительно определённая квадратичная форма. Тогда
    $$ \forall dx \in S_1^{m-1}(0) \hookrightarrow d^2_{x^0} f(dx) > 0, \ f(dx)\text{~---~непрерывная функция от } dx.$$

    Единичная сфера $S_1^{m-1}$~---~компакт $\Longrightarrow$ $\exists$ минимум $d^2_{x^0} f(dx)$ на единичной сфере. Обозначим его $m$. \\
    $$ \forall dx \in S_1^{m-1}(0) \ \hookrightarrow \ d^2_{x^0} f(dx) \geq m$$
    $$ \Longrightarrow \forall dx \neq 0$$
    $$ d^2_{x^0} f(dx) = \| dx \|^2  \sum\limits_{i=1}^m \sum\limits_{j=1}^m \dfrac{\partial^2 f}{\partial x_i \partial x_j} (x^0) \dfrac{dx_i}{\|dx\|} \dfrac{dx_j}{\|dx\|} \geq m \|dx\|^2 $$
    $$ \Longrightarrow \exists \widetilde{\delta} > 0, \text{ такое что } \left|o(\| x - x^0 \|^2)\right| \leq \dfrac{1}{4} \left| d^2_{x^0} f(dx) \right| \ \forall x \in B_{\widetilde{\delta}} (x^0)$$
    $$ \Longrightarrow \forall x \in B_{\widetilde{\delta}}(x^0) \hookrightarrow f(x) \geq f(x^0) + \dfrac{d^2 f(x^0)}{2}(dx) - \left|o(\| x - x^0 \|^2)\right| $$
    $$ \Longrightarrow f(x^0) \geq f(x^0) + \dfrac{m}{4} \| x - x^0 \|^2$$
    $$ \Longrightarrow x^0 \text{~---~ точка строгого локального минимума}.$$

    \item Мгновенно следует из первого заменой $f$ на $-f$.

    \item Пусть $d^2_{x^0} f(dx)$~---~знаконеопредеделена. Тогда $\exists dx^2$, такой что $d^1_{x^0} f(dx^1) > 0$, и \\
    $\exists dx^2 \neq 0$, такой что $d^2_{x^0} f(dx^2) < 0$.
    $$ f\left(x^0 + t dx^1\right) - f(x^0) = \dfrac{t^2}{2} d^2_{x^0} f(dx^1) + o(t^2), \ t \to 0, $$
    $$ f\left(x^0 + t dx^2\right) - f(x^0) = \dfrac{t^2}{2} d^2_{x^0} f(dx^2) + o(t^2), \ t \to 0, $$
    $$ \Longrightarrow \exists \delta_1 > 0 \ : \forall t \in (0, \delta_1) \hookrightarrow f(x^0 + t dx^1) - f(x^0) > 0, $$
    $$ \exists \delta_2 > 0 \ : \forall t \in (0, \delta_2) \hookrightarrow f(x^0 + t dx^2) - f(x^0) < 0. $$
    $x^0 + t dx^1 \in B_{\delta}(x^0),$ \\
    $x^0 + t dx^2 \in B_{\delta} (x^0) $ \\ 
    Можем выбрать пару $t_1, \ t_2$, для которых приращения будут иметь разный знак, то есть
    $$\forall \delta > 0 \ \exists t_1(\delta) = min\left\{ \dfrac{\delta_1}{2}, \dfrac{\delta}{2 \| dx^1\|} \right\}$$
    $$\forall \delta > 0 \ \exists t_2(\delta) = min\left\{ \dfrac{\delta_2}{2}, \dfrac{\delta}{2 \| dx^2\|} \right\}$$

    $$ \begin{cases}
        f(x^0 + t_1(\delta) dx^1) - f(x^0) > 0, \ x^0 + t_1(\delta) dx^1 \in B_{\delta}(x^0) \\
        f(x^0 + t_2(\delta) dx^2) - f(x^0) < 0, \ x^0 + t_2(\delta) dx^2 \in B_{\delta}(x^0)
    \end{cases} $$

    $\Longrightarrow$ $x^0$ не является точкой локального экстремума функции $f$.
    
\end{enumerate}

\end{proof}
