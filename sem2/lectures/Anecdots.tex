\newpage

\section{Сборник Цитат от Маэстро}
Здесь собраны цитаты, истории да и просто приятные моменты с лекций Александра Ивановича Тюленева

\begin{enumerate}
    \item Эта теорема как вопрос:
    Какой у вас телефон?
    Мобильный

    \item Мы будем заниматься по бразильнской системе:
    Я расскажу сколько захочу,
    Вы поймете столько сколько сможете

    \item На пустом множестве анализ прекрасен, но бесполезен

    \item Как здесь(Дифферинциал k порядка) нормальные пацаны не пишут, 
    Они используют мультииндексовые обозначения

    \item Как вы это называете зависит от вашей тусовки:
    Если вы в функционално-аналитической тусовке, то для вас это линейный оператор
    Если вы в геометрической, то линейное отображение
    Если в алгебраической, то гомоморфизм
    
    \item (4 лекция 2 семестр, на паре дифф формы и линейные операторы)
    Подробно вы будете изучать это на 3 курсе, если доучитесь конечно...

    Обычно эта поправка вызывает больше смеха, вы либо уверены что все доучитесь до 3 курса, а может наоборот

    \item (Пришли к производной по направлению через операторы, и вопрос что это такое) Что, кино не смотрите со мной в главной роли?

    \item Лекция 1 48:55 История с Бмкой

    \item Я вам рассказывал военный анекдот про танки?
    Проходит планерка у военных. Генерал проводит совещание и говорит:
    - Товарищи офицеры, к нам в роту прибыло 28 танков. Нужно разделить их по 13 танков в 7 рот.
    Все записывают, ни у кого вопросов не возникает, только с последней парты младший лейтенант робко тянет руку и говорит:\\
    -Товарищ лейтенант, ну у меня никак не получается разделить 28 танков на 13 танков по 7 рот. Я неопытный человек, объясните мне пожалуйста\\
    - Ну как же, очевидно же, вот подполковник, объясните ему, там в столбик, все легко же. \\
    - Ну давай в столбик поделим, 28 делить на 7. Восемь поделить нацело на 7 это один. Записывает первую цифру 1. Остается 21, поделить на 7 получается 3. В итоге 13. \\
    Лейтенант загрузился, сидит думает и говорит, я вот на калькуляторе посчитал, никак не получается у меня. \\
    - Ему говорят, ну ты вот глупый что-ли, умножением проверь. 13 на 7 сколько будет: 3 на 7 = 21. Плюс 7 на 1, получается 28. И отстань от меня!\\

    После этого он опять говорит, ну я все равно не понимаю. Все уже устали ему объянсять и говорят, ну капитан, объясни хоть ты ему. Он совсем безнадежный, не понимает арифметики армейской. Капит говорит, я тебе сложением покажу (в столбик записано 7 раз по 13) и считаем: 3+3+3+3+3+3+3 = 21. Теперь считаем единцы 21, 22, 23, 24, 25, 26, 27, 28. Ну вот!(Закадровый смех Платона). Лейтинант говорит:\\
    - Я уже на все согласен, я все понял, 28 танков, 7 рот, по 13 танков, все. Но вы объясните мне, внутри роты как мне эти танки распределить? \\

    Ну очень просто, 3 танка по взводам расставляешь, а 1 себе!\\
    Вот такой армейский юмор)
    

    \item Эта штука принадлежит лучу от нуля до светлого будущего

    \item Обратите внимание, за эту лекцию мы не доказали ни одной теоремы. Я вас обучаю так называемому языку, на котором мы будем говорить.

    \item Что-то у вас взгляд отсутствующий. Вы поняли че я сказал?

    \item Специально для этого билета на экзамене должен дежурить наряд скорой помощи и психушки. Только им эту теорему не рассказывайте. 

    \item Зарапортовался...

    \item Вы попали на канал Тюленев ТВ. Сейчас у нас тут будет кино. Цикл передач: Матанализ

    \item Даже среди ваш поток можно разбить на отношение, кто с кем общается. Это не будет отношением эквивалетности. Отсуствие симметрии - это безответная любовь. Отсутсвие рефлексии - это сигнал о дурдоме. Отсутствие транзитивности Ну так бывает) 

    \item (Это надо видеть 1:10:00 19 лекция) Вот так хоп. Дилинь-Дилинь-Дилинь хоп и разошелся. Потом Парам-Парам-Парам и сошелся

    \item Вопросы, Questions, Remarks?

    \item (После многомерного анализ пришли к числовам рядам) Ну должны же быть какие-то перепады в жизни. Не все же плохо должно быть поряд. Вообще у нас с вами нисходящая дуга. Мы с вами на многообразиях были... а сейчас у нас какие-то последовательности числовые, какой-то детский сад
\end{enumerate}