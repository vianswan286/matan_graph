\begin{definition}
    Пусть $\Omega \subset \R^n$~---~открытое непустое множество, $k \in \N$. Назовём $\DIF^{k} (\Omega)$ \textit{множество всех $k$ раз дифференцируемых в каждой точке $\Omega$ функций}.
\end{definition}

\begin{note}
    Заметим, что это линейное пространство.
\end{note}

\begin{definition}
    Пусть $\Omega \subset \R^n$~---~открытое непустое множество, $F$: $\Omega \mapsto \R$ является $k$ раз дифференцируемой в точке $x^{0} \in \Omega$. \textit{Дифференциалом} $k$-го порядка назовём дифференциал от дифференциала $(k-1)$-го порядка, то есть $\forall k \in \N$ $\displaystyle d_{x^{0}}^{k} [F] = d \left[d_{x^{0}}^{k-1} F \right]$, при этом $d_{x_i}$ считаются фиксированными % в случае $(k-1)$.

    \textit{Дифференциалом нулевого порядка} будем считать саму функцию.
\end{definition}

\begin{lemma}
    Пусть $F$: $\Omega \mapsto \R$ $k$ раз дифференцируемо в точке $x^{0} \in G$. Тогда
    $$\displaystyle d_{x^0}^{k} F (dx) = \sum\limits_{i_1 = 1}^{n} \ldots \sum\limits_{i_k = 1}^{n} \frac{\partial^{k} F (x^0)}{\partial x_{i_k} \ldots \partial x_{i_1}} dx_{i_1} \ldots dx_{i_k}.$$
\end{lemma}

\begin{proof}
    Доказательство по индукции.

    База: $k = 1$ получаем выражения для записи первого дифференциала.

    Шаг: пусть верно для $k = l$, докажем для $l + 1$.

    $$d_{x^0}^{l + 1} F (dx) = d_{x^{0}} \left( d_{x^{0}}^{l} F (dx)\right) = d_{x^{0}} \left( \sum\limits_{i_1 = 1}^{n} \ldots \sum\limits_{i_l = 1}^{n} \frac{\partial^{l} F (x^0)}{\partial x_{i_l} \ldots \partial x_{i_1}} dx_{i_1} \ldots dx_{i_l}\right) \quad (*)$$
    По линейности дифференциала получаем $(*) = \sum\limits_{i_1 = 1}^{n} \ldots \sum\limits_{i_l = 1}^{n} d_{x^{0}} \left(\frac{\partial^{l} F (x^0)}{\partial x_{i_l} \ldots \partial x_{i_1}} dx_{i_1} \ldots dx_{i_l}\right) = $
    $= \sum\limits_{i_1 = 1}^{n} \ldots \sum\limits_{i_l = 1}^{n} \sum\limits_{i_{l+1} = 1}^{n}\frac{\partial^{l+1} F (x^0)}{\partial x_{i_{l+1}}\partial x_{i_l} \ldots \partial x_{i_1}} dx_{i_1} \ldots dx_{i_l} dx_{i_{l + 1}}$, чего мы и хотели, так как порядок суммирования мы можем менять.
\end{proof}

\begin{note}
    Если все частные производные $k$-го порядка оказались непрерывны в какой-то точке $x^{0} \in \Omega$, то 
    $$ \cfrac{\partial^{k} F (x^{0})}{\partial x_{i_1} \ldots \partial x_{i_k}} = \cfrac{\partial^{k} F (x^{0})}{\partial x_{\sigma(i_1)} \ldots \partial x_{\sigma(i_k)}},$$
    где $\sigma$~---~произвольная перестановка индексов $i_{1}, \ldots, i_{k}$. Тогда выражение для $d_{x^0}^{k} F (dx)$ можно переписать в более компактном виде.
\end{note}


\begin{definition}
    Пусть $\alpha \in \N_{0}^{n}$, то есть, $\alpha = (\alpha_1, \alpha_2, \dots \alpha_n)$, где $\alpha_i \in \N_{0}, i \in \{1, 2, \dots n\}$. Также определим $|\alpha| = \sum_{i = 1}^n \alpha_i$. Тогда будем называть мультииндексным обозначением
    \[D^\alpha F(x^0) := \frac{\partial^{|\alpha|} F}{\partial x_1^{\alpha_1} \cdot \partial x_2^{\alpha_2} \dots \partial x_n^{\alpha_n}}\]
\end{definition}

\begin{note} 
    Теперь можно переписать $k$-ый дифференциал в более компактном виде.
    Пусть $F$ имеет все производные $k$-ого порядка непрерывные в точке $x^{0}$. Тогда \[d_{x^0}^k F(d x) = \sum_{|\alpha| = k} C_{\alpha} D^{\alpha} F(x^0) dx^{\alpha}\]
    где $C_{\alpha} = \frac{k!}{\alpha_1! \cdot \alpha_2! \dots \alpha_n!}$ и $dx^{\alpha} = (dx_1)^{\alpha_1} \cdot (dx_2)^{\alpha_2} \dots (dx_n)^{\alpha_n}$.
    
\end{note}

\newpage

\subsection{Линейные операторы и их нормы}

\begin{definition}
    Пусть $E_1$, $E_2$~---~линейные пространства. 
    Отображение $A$: $E_1 \mapsto E_2$~---~\textit{линейный оператор} или \textit{линейное отображение} (или \textit{гомоморфизм}) из $E_1$ в $E_2$, \\
    если выполнено \[A(\alpha x + \beta y) = \alpha A(x) + \beta A(y), \hspace{0.5cm} \forall \alpha, \beta \in \R \hspace{0.2cm} \forall x, y \in E_{1}.\]
\end{definition}

\begin{note}
    Множество всех линейных операторов из $E_1$ в $E_2$ можно само наделить структурой линейного пространства. Пусть $A$: $E_1 \mapsto E_2$~---~линейный оператор и $B$: $E_1 \mapsto E_2$~---~линейный оператор. Тогда введём 
    \begin{table}[h]
    \centering
    \begin{tabular}{ll}
    $(A+B) (x) := A(x) + B(x)$ & $\forall x \in E_1$; \\
    $(\alpha A)(x) := \alpha \cdot A(x)$ & $\forall \alpha \in \R, \forall x \in E_1$.
    \end{tabular}
    \end{table}
\end{note}

\begin{definition}
    Пусть $E_1$, $E_2$~---~линейные нормированные пространства, \\ $A$: $E_1 \mapsto E_2 $~---~линейный оператор. Определим его \textit{норму} как 
    \[ \|A\| := \sup_{x \neq 0}{\frac{\|A(x)\|_{E_2}}{\|x\|_{E_1}}} \in [0, +\infty]. \]

    Если $\|A\| < +\infty$, то $A$ называется \textit{ограниченным линейным оператором} из $E_1$ в $E_2$.
\end{definition}

\begin{note}
    Не очень ясно как считать супремум по всем $x \neq 0$. Преобразуем $\frac{\|A(x)\|_{E_2}}{\|x\|_{E_1}}$ следующим образом, используя свойство нормы (занесём неотрицательный на скаляр под нее). Далее, используя линейность оператора можно занести скаляр в $A(x)$. Таким образом: \\
    \[ \frac{\|A(x)\|_{E_2}}{\|x\|_{E_1}} = \bigg\|\frac{1}{\|x\|_{E_1}} \cdot A(x) \bigg\|_{E_2} = \bigg\|A \left( \frac{x}{\|x\|_{E_1}} \right) \bigg\|_{E_2} \]
    Теперь же заметим, что $\bigg\| \frac{x}{\|x\|_{E_1}} \bigg\|_{E_1} = 1 \Longrightarrow $ \\ $\Longrightarrow $ если $x$ пробегает все $\left(E_1 \setminus 0\right)$, то $ \frac{x}{\|x\|_{E_1}} $ пробежит единичную сферу в $E_1$.

    \[ \sup_{x \neq 0} \frac{\|A(x)\|_{E_2}}{\|x\|_{E_1}} = \sup_{x \neq 0} \bigg\|\frac{1}{\|x\|_{E_1}} \cdot A(x) \bigg\|_{E_2} = \sup_{x \neq 0} \bigg\|A \left( \frac{x}{\|x\|_{E_1}} \right) \bigg\|_{E_2} = \sup_{x \in S_1^{E_2}(0)}{\|A(x)\|}\] 

\end{note}

\begin{definition}
    Пусть $A$~---~линейный оператор. \textit{Ядром} линейного оператора называется множество элементов которые переходят в $0$ и обозначается $\Ker A$.
\end{definition}

\begin{note}
    Ядро линейного оператора образует \textit{линейное подпространство}. \\ 
    Если подпространство совпадает с линейным пространством, то такой оператор называется \textit{нулевым}.
\end{note}

\begin{proposition}
    Пусть $A$~---~линейный оператор. $A$: $E_1 \mapsto E_2$~---~нулевой оператор $\Leftrightarrow \| A\| = 0$.
\end{proposition}

\begin{proof}
    Заметим, что если $A \equiv 0$, то $\| A \| = 0$.

    С другой стороны если $\| A \| = 0$, то $\| A(x) \| = \cfrac{\| A(x) \|}{\| x \|} \| x \| \leq \| A \| \| x \| \Rightarrow \| A (x) \| = 0 \Rightarrow A (x) = 0 \Rightarrow \text{Ker}A = E_1 \Rightarrow A$~---~тождественно нулевой оператор.
\end{proof}

\begin{note}
    Тождественно нулевой оператор играет роль нуля в пространстве всех линейных операторов.
\end{note}

\begin{theorem}
    Пусть $E_1$, $E_2$~---~линейные нормированные пространства. Множество всех ограниченных линейных операторов из $E_1$ в $E_2$ имеет естественную структуру линейного нормированного пространства. \\
    В обозначениях современного функционального анализа: \\
    $\mathcal{L}(E_1, E_2)$~---~линейное нормированное пространство всех ограниченных линейных операторов из $E_1$ в $E_2$.
\end{theorem}

\begin{proof}
    Проверим каждое из условий линейного нормированного пространства:   
    \begin{enumerate} 
        \item Оператор нулевой тогда и только тогда, когда его норма равна нулю. Было доказано только что.
        \item $\| \alpha A \| = \sup_{x \neq 0}{\frac{\|\alpha A(x)\|}{\|x\|}} = |\alpha| \sup_{x \neq 0}{\frac{\|A(x)\|}{\|x\|}} = |\alpha| \cdot \|A\|$.
        \item Проверим же теперь неравенство треугольника (с учётом $\| A\|, \| B\| < +\infty$):\\
        $\| A + B \| = \sup_{x \neq 0}{\frac{\| A(x) + B(x) \|}{\|x\|}} \leq \sup_{x \neq 0}{\frac{\|A(x)\| + \|B(x)\|}{\|x\|}} \leq \sup_{x \neq 0}{\frac{\|A(x)\|}{\|x\|}} + \sup_{x \neq 0}{\frac{\|B(x)\|}{\|x\|}} = \\ = \|A\| + \|B\|$.
    \end{enumerate}
\end{proof}

\begin{note}
    Неограниченные операторы бывают только в бесконечномерных пространствах.
\end{note}

\begin{note}
    Геометрический смысл нормы оператора: \\
    $\|A\| = \sup_{x \neq 0}{\frac{\|A\|_{E_2}}{\|x\|_{E_1}}}$~---~<<максимальное>> искажение длины при линейном отображении.
\end{note}

\begin{lemma}
    Пусть $A \in \mathcal{L}(E_1, E_2)$, тогда $\forall x \in E_1$ справедливо неравенство:
    \[\|A(x)\|_{E_2} \leq \|A\| \cdot \|x\|_{E_1}\]
\end{lemma}

\begin{proof}
    При $x = 0$, обе части обращаются в ноль и доказывать нечего.

    При $x \neq 0$, $\|A(x)\|_{E_2} = \frac{\|A(x)\|_{E_2}}{\|x\|_{E_1}} \cdot \|x\|_{E_1} \leq \sup_{x \neq 0}{\frac{\|A(x)\|_{E_2}}{\|x\|_{E_1}}} \cdot \|x\|_{E_1} = \|A\| \cdot \|x\|_{E_1} $.
\end{proof}

Рассмотрим пример неограниченного линейного оператора.

\begin{example}
    Пусть $E_1$~---~линейное пространство всех непрерывно дифференцируемых на $[0, 1]$ функций с нормой: 
    \[\|f \| = \sup_{x \in [0; 1]}{|f(x)|} = \max_{x \in [0; 1]}{|f(x)|}.\] 
    Пусть $E_2$~---~линейное пространство всех непрерывных на $[0; 1]$ функций с нормой $\|g\| = \max_{x \in [0; 1]}{|g(x)|}$.

    Пусть $A$: $\frac{d}{dx}$~---~оператор дифференцирования функций из $E_1$ в $E_2$.
    
    Посчитаем его норму по определению:

    \[\displaystyle \| A \| = \sup_{f \neq 0}{\frac{\underset{x \in [0; 1]}{\max}{\left|\frac{df}{dx}\right|}}{\underset{x \in [0; 1]}{\max}{|f(x)|}}} \geq \sup_{n \in \N}{\frac{\underset{x \in [0; 1]}{\max}\left|{\frac{df_n}{dx}(x)}\right|}{\underset{x \in [0; 1]}{\max}{|f_n(x)|}}} = \sup_{n \in N} n = +\infty.\]

    где $f_n(x) = sin(nx)$, $f'_n(0) = n$, $\max_{x \in [0; 1]}{|f'_n(x)|} = n$, $\max_{x \in [0; 1]}{|f_n(x)|} = 1$.
\end{example}

\begin{theorem}
    Если $E_1 \subset \R^n$, $E_2 \subset \R^m$~---~линейные нормированные пространства с евклидовой нормой (евклидовы пространства), то любой линейный оператор $A$: $E_1 \mapsto E_2$~---~ограниченный.
\end{theorem}

\begin{proof}
    Пусть $\{ e_1, e_2, \dots e_n \}$~---~базис в $E_1$. Тогда любому $A$ соответствует матрица оператора в этом базисе:

    $$ A = \begin{pmatrix}
  a_{11}& \ldots & a_{1n}\\
  \vdots & \ddots & \vdots \\
  a_{m1}& \ldots & a_{mn}
\end{pmatrix}$$

    Пусть $m$~---~размерность $E_2$, $A(e_i)$~---~$i$-ый столбец матрицы $A$. Тогда в $E_1$ имеет место разложение $x = \sum_{i = 1}^n x_i \cdot e_i$.
    $$\| A x \| = \left\| A \left( \sum_{i = 1}^{n} x_{i} e_{i}\right)\right\| = \left\| \sum_{i = 1}^n x_i \cdot A(e_i) \right\| \leq \sum_{i = 1}^n \| x_i \cdot A(e_i) \| = \sum_{i = 1}^n |x_i| \|A(e_i)\| \leq$$
    $$\leq \text{по неравеству Коши-Буняковского} \leq \left (\sum_{i = 1}^n {x_i^2} \right)^{1/2} \cdot \left(\sum_{i = 1}^n {\left\|A e_i\right\|}^2 \right)^{1/2}.$$

     Заметим, что $\left(\sum_{i = 1}^n {x_i^2} \right)^{1/2} = \| x \| < + \infty$ , также $\|A\| \leq \sum_{i = 1}^n {\|A e_i\|}^2 =\sum_{i = 1}^n \sum_{i = 1}^m a_{i, j}^2 < + \infty $

     Итого получаем, что 
     \[ \|Ax\| \leq \left(\sum_{i = 1}^n \sum_{i = 1}^m a_{i, j}^2\right)^{1/2} \cdot \|x\| < +\infty \]
\end{proof}

\begin{definition}
    Пусть $E_1, E_2, E_3$~---~линейные пространства, $A$: $E_1 \mapsto E_2$~---~линейный оператор, $B$: $E_2 \mapsto E_3$~---~линейный оператор.

    Тогда \textit{композицию (или произведение)} операторов $A \text{ и } B$ называют оператор $B \circ A := BA$~---~композиция отображений $A$ и $B$.
\end{definition}

\begin{note}
    Заметим, что $BA$~---~ линейный оператор из $E_1$ в $E_3$:
    \[(BA)(\alpha x + \beta y) = B( A(\alpha x + \beta y) ) = B( \alpha A(x) + \beta A(y)) = \alpha (BA)(x) + \beta (BA)(y).\]
\end{note}

\begin{lemma}
    Пусть $E_1, E_2, E_3$~---~линейные нормированные пространства. \\ Пусть $A \in \mathcal{L}(E_1, E_2), B \in \mathcal{L}(E_2, E_3)$. Тогда $(BA) \in \mathcal{L}(E_1, E_3)$ и при этом $\|BA\| \leq \|B\| \cdot \|A\|$.
\end{lemma}

\begin{proof}
    Вспомним свойство: $\| Ax \| \leq \|A\| \cdot \| x \| \forall x$
    \[\|(BA)(x)\|_{E_3} \leq \|B\| \cdot \|A(x)\|_{E_2} \leq \|B\| \cdot \|A\| \cdot \|x\|_{E_1}\]
    $\Rightarrow$ если $x \neq 0$, то 
    \[\frac{\|(BA)(x)\|_{E_3}}{\|x\|_{E_1}} \leq \|B\| \cdot \|A\| \Rightarrow \|BA\| \leq \|B\| \cdot \|A\|\]
\end{proof}

\begin{remark}
    Даже если $E_1 = E_2 = E_3$, $A$ и $B$ могут не коммутировать, то есть $A \cdot B \neq B \cdot A$.

    Например, $A = \sum_{i = 1}^n \lambda_i \cdot \dfrac{\partial }{\partial x_i}$, где $\lambda = (\lambda_1, \lambda_2, \ldots, \lambda_n) \neq 0$. Причем $A$ называется дифференциальный оператор.

    $A$: $\DIF^k(\Omega) \mapsto \DIF^{k-1}(\Omega)$, $k \in \N.$ 
\end{remark}

\begin{note}
    Пусть $F \in \DIF^1(\Omega)$.

    Тогда $(AF)(x^0) = \left(\sum_{i = 1}^n \lambda_i \dfrac{\partial}{\partial x_i} \right) F(x^0) = \sum_{i = 1}^n \lambda_i \cdot \dfrac{\partial F}{\partial x_i}(x^0) = \left(\dfrac{\partial F}{\partial \lambda}\right)(x^0) = \langle grad F(x^0), \lambda \rangle$
    Если вместо $\lambda_i$ взять $dx_i$, получится выражение для $d_{x^0}F(dx)$.
\end{note}
