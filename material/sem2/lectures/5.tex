\subsection{Формулы Тейлора с остаточными членами в форме Лагранжа и Пеано}
\begin{note}
    Считаем $dx = (dx_1, \ldots dx_n)$~---~фиксированным вектором;
    \[A = dx_1 \dfrac{\partial}{\partial x_{1}} + \ldots + dx_n \dfrac{\partial}{\partial x_{n}}\]~---~линейный оператор дифференцирования.
    \[A[f](x) = d_x f(dx).\]
\end{note}

\begin{lemma}
    Пусть $\Omega \subset \R^n$, $\Omega \neq \varnothing$~---~открыто, $f$: $\Omega \rightarrow \R$~---~$k-$раз дифференцируемо в любой точке $x^{0} \in \Omega$. Тогда $$\forall dx \in \R^n \hookrightarrow A^k[f](x^0)(dx) = d^k_{x_0}[f](dx).$$
\end{lemma}
\begin{proof}
    Доказательство очевидно, предполагается читателю в качестве упражнения (спойлер: по индукции).
\end{proof}
\begin{theorem}[Формула Тейлора с остаточным членом в форме Лагранжа] $\newline$
    Пусть $m \in \N_0$, $\Omega \subset \R^n$, $\Omega \neq \varnothing$~---~открыто, $f$: $\Omega \rightarrow \R$ и $f \in \DIF^{m + 1}(B_{\delta}(x^{0}))$ при каком-то $\delta > 0$. Тогда $\forall x \in B_{\delta}(x^{0})$ $\exists \theta \in (0, 1)$: справедлива следующая формула
    \[f(x) = f(x^0) + \sum\limits_{k = 1}^{m}\frac{d^k_{x^0}(dx)}{k!} + \frac{d^{m + 1}_{x^{\theta}}(dx)}{(m + 1)!}, \text{ где }dx = x - x^0, x^{\theta} = x^0 + \theta(x - x^0).\]
\end{theorem}
\begin{proof}
    Зафиксируем точку $x \in B_{\delta}(x^0)$. Пусть $x^t = x^0 + t(x - x^0)$, где $t \in [0, 1]$. Рассмотрим функцию $\phi(t) = f(x^t) = f\left(x^0 + t(x - x^0)\right)$. Заметим, что \[\dfrac{d\phi}{dt} (t) = \sum\limits_{i = 1}^n \dfrac{\partial f}{\partial x_i}(x^t) \cdot (x_i - x^0_i) = d_{x^t}f(dx)\]
    Тогда $\forall s \in \{1, \ldots, m + 1\}$ будет выполнено следующее равенство: 
    \[\dfrac{d^s \phi}{d t^s}(t) = d^s_{x^t} f(dx).\]
    Докажем это по индукции. База была показана выше, докажем переход от $s$ к $s + 1$:
    \[\dfrac{d^{s + 1}}{dt^{s + 1}} \phi(t) = \dfrac{d}{dt}(d^s_{x^t} f(dx)) = \dfrac{d}{dt} \left(\sum\limits_{i_1 = 1}^n\sum\limits_{i_2 = 1}^{n} \ldots \sum\limits_{i_s = 1}^n \dfrac{\partial^sf}{\partial x_{i_s} \ldots \partial x_{i_1}}(x^t)dx_{i_1} \ldots dx_{i_s}\right) =\]
    \[= \sum\limits_{i_1 = 1}^n \ldots \sum\limits_{i_{s + 1} = 1}^n \dfrac{\partial^{s + 1}f}{\partial x_{i_{s+1}} \ldots \partial x_{i_1}}dx_{i_1}\ldots dx_{i_{s+1}} = d^{s + 1}_{x^t} f(dx)\]
    Значит, к $\phi$ можно применить одномерную формулу Тейлора с остаточным членом в форме Лагранжа:
    \[\phi(1) = \phi(0) + \sum\limits_{k = 1}^m \frac{1}{k!} \dfrac{d^k \phi}{dt^k}(0) + \frac{1}{(m + 1)!} \dfrac{d^{m + 1}\phi}{dt^{m + 1}}(\theta) \cdot 1\]
    \[f(x) = f(x^0) + \sum\limits_{i = 1}^m \frac{1}{i!} d_{x^0}^if(dx) + \frac{1}{(m + 1)!}d_{x^\theta}^{m + 1}f(dx)\]
\end{proof}
\begin{theorem}[Формула Тейлора с остаточным членом в форме Пеано]
    Пусть $f$: $\Omega \rightarrow \R, m \in \N$ и $\exists \delta > 0$: $f \in C^m(B_\delta(x^0))$ (f~---~$m$-гладкая). Тогда $\forall x \in B_{\delta}(x^0)$: 
    \[f(x) = f(x^0) + \sum\limits_{i = 1}^n \frac{1}{i!}d^k_{x^0} f(dx) + o(||x - x^0||^m), x \rightarrow x^0\]
\end{theorem}
\begin{proof}
    Все частные производные порядка $m$~---~непрерывны в $B_{\delta}(x^0)$, значит, по достаточному условию дифференцируемости, все частные производные $m - 1$ порядка являются дифференцируемыми функциями в этом шаре. $\Rightarrow f \in \DIF^m(B_{\delta}(x^m) \Rightarrow$ по формуле Тейлора с остаточным членом в форме Лагранжа:
    \[\forall x \in B_{\delta}(x^0) \ \exists \theta \in (0, 1):\]
    \[f(x) = f(x^0) + \sum\limits_{k = 1}^{m - 1} \dfrac{d_{x^0}^k f(dx)}{k!} + \dfrac{1}{m!} d_{x^\theta}^m f(dx) = (*), x^\theta = x^0 + \theta(x - x^0)\]

Добавим и вычтем m дифференциал в точке $x^0$

    \[(*) = f(x^0) + \sum\limits_{k = 1}^m \dfrac{d_{x^0}^k f(dx)}{k!} + R, R = \dfrac{1}{m!}[d_{x^\theta}^m f(dx) - d_{x^0}^m f(dx)]\]
    Покажем, что $R = o(||x - x^0||), x \rightarrow x^0$. Тогда по неравенству треугольника
    \[|R| \leq \dfrac{n^m}{m!}\sum\limits_{i_1 = 1}^n \cdot \ldots \cdot \sum\limits_{i_m = 1}^n| \dfrac{\partial^m f}{\partial x_{i_m} \cdot \ldots \cdot \partial x_{i_1}}(x^\theta) - \dfrac{\partial^m f}{\partial x_{i_m} \cdot \ldots \cdot \partial x_{i_1}}(x^0)|
    \cdot |dx_{i_1}| \cdot \ldots \cdot |dx_{i_n}| \leq (*)\]
    В то же время, из непрерывности частных производных $m-$го порядка следует
    \[\varepsilon_{x^0}(x) = \max\limits_{i_1, \ldots, i_m} |\dfrac{\partial^m f(x^{\theta})}{\partial x_{i_m} \cdot \ldots \cdot \partial x_{i_1}} - \dfrac{\partial^m f(x^0)}{\partial x_{i_m} \cdot \ldots \cdot \partial x_{i_1}}| \rightarrow 0, x \rightarrow x^0\]
    Поскольку $x^\theta \xrightarrow{x \rightarrow x^0} x^0 \Rightarrow |dx_{i_1}\cdot \ldots \cdot dx_{i_m}| \leq ||x - x^\theta||^m$, поскольку $|dx_{i_j}| = |x_{i_j}^0 - x_{i_j}| = \sqrt{|x^0_{i_j} - x_{i_j}|^2} \leq ||x - x^0||$. 
    \[(*) \leq \dfrac{m^n}{m!} \varepsilon_{x^0} (x) ||x - x^0||^m = o(||x - x^0||^m), \ x \rightarrow x^{0}.\]
\end{proof}
\begin{note}
    Справедлива теорема о единственности: если $f \in C^m(B_\delta(x^0)) \Rightarrow f(x) = P_m(dx) + o(||x - x^0||^m), x \rightarrow x^0 \Rightarrow P_m(dx)$ совпадает $\sum\limits_{k = 0}^m \dfrac{1}{k!}  d_{x^0}^k f(dx)$, то есть с полиномом Тейлора.
\end{note}


\begin{note} 
Заметим, что в формуле Тейлора с остаточным членом в форме Лагранжа, мы требуем $f \in \DIF^{m + 1}(B_{\delta}(x^{0}))$. А с остаточным членом в форме $f \in C^m(B_\delta(x^0))$
\end{note}

\newpage
\section{Теорема о неявной функции}

\subsection{Теоремы об открытом отображении и локальном диффеоморфизме}
\begin{lemma}
    Пусть $A \in \mathcal{L}(\R^m, \R^m)$,  $A$~---~обратимое. Тогда $\exists c > 0$: $\|A(x) - A(y)\| \geq c\|x - y\|$.
\end{lemma}
\begin{proof}
    Так как $A$ обратимо, то $\exists A^{-1} \Rightarrow x = A^{-1}(A(x)) \Rightarrow \|x|| = \|A^{-1}(A(x))\| \leq \|A^{-1}\|\|A(x)\| \Rightarrow \|A(x)\| \geq \dfrac{1}{\|A^{-1}\|} ||x\| \Rightarrow$ в силу линейности отображения $$\forall x, y \in \R^m\text{: }\|A(x - y)\| = \|A(x) - A(y)\| \geq \dfrac{1}{\|A^{-1}\|}\|x - y\|.$$
\end{proof}

\begin{note}
    Мы научились ограничивать сверху растяжение векторов линейным оператором (Норма оператора). Но оказывается, если оператор обратим, то он растягивает векторы не менее чем на определенную константу (один поделить на норму обратного оператора)
\end{note}
 
\begin{theorem}
\hypertarget{lecture_6_geom_theorem} \ % Почему-то без слеша происходит непонятная ошибка компиляции латеха. UPD: предположу, что hypertarget пытается сослаться на следующий за ним символ, а на русскую "П" он ссылаться не умеет
    Пусть $F$: $\Omega \rightarrow \R^m$, $\Omega \subset \R^m$~---~дифференцируемо в точке $x^0 \in \Omega$ и матрица Якоби отображения, $DF(x^0)$,~---~обратима. Тогда $\exists \delta > 0$ и $\exists c > 0$: $\forall x \in \overline{B}_\delta(x^0) \hookrightarrow \|F(x) - F(x^0)\| \geq c \|x - x^{0}\|$.
\end{theorem}
\begin{proof}
    Поскольку $F$~---~дифференцируемо в точке $x^0 \Rightarrow$
    \[F(x) = F(x^0) + DF(x^0)(x - x^0) + \varepsilon_{x^0}\|x - x^0\|\]
    По предыдущей лемме, $ \|DF(x^0)(x - x^0)\| \geq 2C \cdot \|x - x^0\|, 2C = \dfrac{1}{\|(DF(x^0))^{-1}\|}$
    Тогда по неравенству треугольника
    (Прим ред.) На лекции это опущено, но идея такая $a = b + c \Longrightarrow |b| = |a - c| \leq |a| + |c| \Longrightarrow |a| \geq |b| - |c|$:
    
    \[\|F(x) - F(x^0)\| \geq \|DF(x^0)(x - x^0)\| - \big{\|} \varepsilon_{x^0}(x)\|x - x^0\|\big{\|} \geq (*)\]
    Но $\big{\|} \varepsilon_{x^0}(x)\| x - x^0 \| \big{\|} \rightarrow 0, x \rightarrow x^0$, значит $\exists \delta > 0$: $\|\varepsilon_{x^0}(x) \| \leq \dfrac{1}{2\|(DF(x^0))^{-1}\|}$. Поэтому
    \[(*) \geq \dfrac{1}{\|(DF(x^0))^{-1}\|} \|x - x^0 \| - \dfrac{\|x - x^0\|}{2\|(DF(x^0))^{-1}\|} = c\|x - x^0\|\]
    И это верно для $\forall x \in B_\delta(x^0)$. Из нестрогости неравенства $\Rightarrow$ справедливость неравенства и для $\overline{B}_\delta(x^0)$. (Небольшое замечание: всё это время мы работаем с $\overline{B}_\delta(x^0) \subset \Omega$)
\end{proof}

\begin{note}
    Отличие теоремы и леммы в том, что лемма для линейного отображение, а в теореме мы доказали уже для произвольного отображения. Идея в том, что дифференцируемость~---~это приближение линейным, с точностью до бесконечно малого, более высокого порядка. Если матрица Якоби обратима, то линейное отображение будет удовлетворять оценке из леммы
\end{note}

\begin{theorem}[Теорема об открытом отображении]
    \ \newline
    Пусть $F: \Omega \rightarrow \R^m$$, \Omega \text{~---~непустое и открытое множество},$
    $ F \in \DIF(\Omega, \R^m)$ и пусть $\forall x^0 \in \Omega \hookrightarrow DF(x^0)$~---~обратима (Якобиан не обращается в 0)
    
    Тогда $F(\Omega)$~---~открыто в $\R^m$
\end{theorem}
\begin{proof}
    (Прим. ред.) Мы хотим доказать открытость по определению. Мы берем произвольную точку, смотрим на куда она переходит, предъявляем такой $r$, что открытый $r$-шар вкладывается в образ, доказывая что для любой точки из этого шара есть прообраз. \\
    
    Фиксируем $x^0 \in \Omega$ и $y^0 = F(x^0)$. Покажем, что $\exists r > 0$ такое что $B_r(y^0) \subset F(\Omega)$: 
    Пользуясь предыдущей теоремой $\exists \delta > 0$ и $c > 0$: $\overline{B}_\delta \subset \Omega$ и $\|F(x) - F(x^0)\| \geq c \|x - x^0\| \forall x \in \overline{B}_\delta(x^0)$. Покажем что $r = \dfrac{c\delta}{2}$ является искомым. Возьмём произвольную $y^* \in B_{r}(y^0)$ и покажем что $\exists x^* \in B_\delta(x^0)$: $F(x^*) = y^*$. Заметим, что
    \[\forall x \in S_\delta (x^0) \hookrightarrow \|F(x) - F(x^0)\| \geq c\delta \]
    \[\|F(x) - y^*\| \geq \|F(x) - y^0\| - \|y^0 - y^*\| \geq c\delta - \|y^0 - y^*\| > \dfrac{c\delta}{2}\]
    Рассмотрим функцию $h(x) = \|F(x) - y^*\|$. \\
,    Тогда, $h(x^0) < r = \dfrac{c\delta}{2}$, а с другой стороны $\forall x \in S_\delta (x^0) \hookrightarrow h(x) > \dfrac{c\delta}{2}$. В то же время, поскольку $h \in C$ как функция от $x$, из компактности $\overline{B}_\delta(x^0)$ следует что $h(x)$ достигает своё наименьшее значение, при этом минимум достигается во внутренней точке, так как, как было показано, на границе $h > \frac{c\delta}{2}$, а минимум не больше $h(x^0)$, которое меньше этой величины. Из неотрицательности нормы следует истинность данного утверждения и для $h^2(x) = \sum\limits_{j = 1}^m(F_j(x) - y_i^*)^2$. Значит $\exists x^* \in B_\delta(x^0)$, в которой $h^2(x)$ принимает минимальное значение. Это внутренняя точка шара $\Rightarrow$ в ней $\grad h = 0$
    \[\dfrac{\partial h^2}{\partial x_i}(x) = \dfrac{\partial}{\partial x_i}(\|F(x) - y^*\|^2) = \]
    \[= \dfrac{\partial}{\partial x_i}(\sum\limits_{j = 1}^m(F_j(x) - y_i^*)^2 = \sum\limits_{j = 1}^m 2(F_j(x) - y^*_j) \dfrac{\partial F_j}{\partial x_i}(x) \Rightarrow \]
    Рассмотрим в точке $x^*$:
    \[\sum\limits_{j = 1}^m 2\dfrac{\partial F_j}{\partial x_i}(x^*) (F_j(x^*) - y_i^*)\Rightarrow\]
    Вспоминая, что j строка матрицы Якоби выглядит как $\text{grad}F_{j} (x^{*}) = \left(\cfrac{\partial F_{j}}{\partial x_{1}} (x^{*}), \ldots, \cfrac{\partial F_{j}}{\partial x_{n}} (x^{*})\right)$
    
    \[\grad h^2(x^*) = 2(\D F(x^*))^T(F(x^*) - y^*) \Rightarrow\]
    Тк матрица Якоби, также как и транспонированая не обращается в ноль по условию
    \[ \Rightarrow F(x^*) - y^* = 0\] 
    Что и требовалось доказать. Мы показали, что $\forall y \in F(\Omega) \ \exists r: B_r(y) \subset F(\Omega) \Longrightarrow \forall y \in F(\Omega) \hookrightarrow y \in Int(F(\Omega))$~---~открытое по определению
\end{proof}

\begin{note}
    Если убрать обратимость, то существует контрпример, образ $(-1, 1)$ для параболы. Интервал переходит в полуинтервал 
\end{note}
%Используй лучше ~---~ для тире (пробелы до и после ~ делать не надо)
%для ... есть \ldots
%также из мелочей можешь пж разделять $$: $$, то есть тип чтобы двоеточие с пробелом было вне формулы (если в двойных $$, то есть тип по центру формула, то \text{: }), так отступ другой оно делает