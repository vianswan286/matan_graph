\newpage


\section{Многообразия и Экстремумы}
\subsection{Многообразия}
\begin{definition}(\underline{Простое} многообразие)
    Пусть $m \in \N$, $1 \leq k \leq m$ и зафиксировано $r \in \N \cup \{+\infty \}$. Будем говорить что множество $\M \in \R^{m}$ является простым многообразием, если:
    \begin{enumerate}
        \item $\exists U (\text{открытое}) \in \R^k$ и $\exists \Phi: U \mapsto \M$ (называемое параметризацией многообразия), которое является \hyperlink{homeomorphism_definition}{гомеоморфизмом} $U$ на $\M$
        \item $\forall t^0 \in U$, $\rank(\D_{t^0}\Phi) = k$. Ранг матрицы $\D_{t^0}\Phi$ максимален всюду в $U$
    \end{enumerate}
    Если при этом $\Phi \in C^{r}(U, \R^m)$, то говорят что параметризация $r$ - гладкая, а само множество называется $k$-мерным, $r$-гладким многообразием\\
    Если $k = m$, то $\M$~---~открытое множество в $\R^{m}$\\
    Если $k=1$, то получается определение $r$-гладкой кривой
\end{definition}

\begin{example}
    Аффинное пространство.
    Пусть $\overline{e_1}, \ldots,\overline{e_k}$~---~линейно независимые вектора в $\R^m$.

    $\mathcal{L} = \{ x\in \R^m: x = \overline{h} + \sum_{t=1}^{k} t_i \cdot \overline{e_i}, t_i \in \R \}$
    $\mathcal{L} - k$-мерное, $\infty$-гладкое многообразие в $\R^m$
\end{example}

\begin{example}
\hypertarget{example3.2}{}



\bold{Было:}
Функция графика отображения. $t \in \R^k$ Пусть $U\subset \R^k,$ и $f$: $U \mapsto R^{m-k}$, ($k<m$)
    Тогда $\Phi = \begin{pmatrix}
  t\\
  f(t)
\end{pmatrix}$~---~многообразие. $\Phi: \R^k \mapsto \R^m$

Докажем по определению:
\begin{enumerate}
    \item Рассмотрим отображение $\overline{\Phi}: \R^m \mapsto \R^m$, которое продолжает отображение: $z \in \R^{m-k}$
    $\overline{\Phi}(t, z) = (t, z + f(t))$ оно непрерывно как комбинация непрерывных
    $\overline{\Phi}^{-1}(t, z) = (t, z - f(t))$

    Если $f$ непрерывно, то $\overline{\Phi}$ и $\overline{\Phi^{-1}}$ непрерывны, значит любое его ограничение на открытое множество непрерывно
\end{enumerate}
Но $\overline{\Phi}(t, 0) = \Phi$, $\Phi^{-1}(t, f(t)) = \Phi$, значит это гомеоморфизм

\bold{Стало:}
Приведём пример простого многообразия в пространстве $\R^m$.
Пусть $U \subset \R^k$ и $f \in C(U, \R^{m - k})$. Тогда
\[
graph f := \{(x, y) : x \in U, y = f(x)\} \text{~---простое k-мерное многообразие.}
\]
В качестве параметризации можно взять $\Phi_{f} : U \mapsto \R^m, \forall t \in U \hookrightarrow \Phi_{f}(t) = (t, f(t))$.
\end{example}

%pure gold
\begin{lemma}
    Многообразие не может иметь параметризаций, определенных в открытых подмножествах пространств различных размерностей.Т ак как иначе они будут диффеморфмы (Те что k - это инвариант многообразия при гомеоморфизмах). Формально:

    \[
    \text{Пусть}
    \begin{cases}
    \exists \Phi \in C^r(U, \R^m), U \subset \R^k \\
    \forall t \in U \hookrightarrow \rank \D_t \Phi = k \\
    \exists \Phi' \in C^r(U', \R^m), U' \subset \R^{k'} \\
    \forall t' \in U' \hookrightarrow \rank \D_{t'} \Phi = k' \\
    \end{cases}
    \]
    Если $\Phi(U) = \Phi '(U') = \M$, то k = k' 
\end{lemma}



\begin{proof}
    Зафиксируем точку $t^0 \in U$. Заметим, что $\rank \D \Phi (t^0) = k$. Выполнены условия теоремы о продолжении до диффеоморфизма. Значит: $\exists \overline{\Phi}: V(t^0) \to W(\Phi(t^0))$, которое является диффеоморфизмом $\overline{\Phi}|_{V(t^0) \cap \R^k} := \Phi$ \\

    (ГФиО стр 173) Рассмотрим композицию $\overline{\Phi}^{-1} \circ \Phi': (\Phi')^{-1} (W(\Phi(t^0)) \cap \M) \to U$. Так как $\Phi^{-1}$ - отображение класса не меньше $C^1$, а $\overline{\Phi}$~---~ диффеоморфизм. Мы получаем $\overline{\Phi}^{-1} \circ \Phi' \in C^1$. \\
    Так как ранг $\D \Phi$ всюду постоянен и равен $k'$, а $(\overline{\Phi})^{-1}$~---~ диффеоморфизм, то $\rank(\overline{\Phi}^{-1} \circ \Phi')(t') = k'$ для любого t' из достаточно малой окрестности \\
    Получаем, что $\overline{\Phi}^{-1} \circ \Phi' \in C^1$ переводящее отношение переводящее открытое подмножество в $U'$ в некоторое открытое подмножество в $U$, ранг которого k $\Longrightarrow k = k'$  
\end{proof}




\begin{definition} (\underline{Многообразие})
    Пусть $m \in \N, \ k \in \{ 1,\dots, m\}$, $r \in \N \cup \{+\infty \}$. Множество $\M \in \R^{m} $ называется $k$-мерным $r$-гладким многообразием, если $\forall p \in \M \ \exists$ окрестность $U(p) \subset \R^{m}$ такой, что $U(p) \cap \M$ является простым $k$-мерным $r$-гладким многообразием. 
\end{definition}

\begin{example}
    Сфера в $\R^{m}$. Не является простым многообразием, но является $m$-мерным $\infty$-гладким многообразием. 
    \begin{proof}
Рассмотрим в $\R^m$ сферу
$$S^{m-1}(R) = \{(x_1,\dots, x_m)\} \in R^{m}: x^2_1 +…+ x^2_m = R^2.$$

Убедимся в том, что это многообразие. У каждой точки $p=(p_1,\dots, p_m)$,
лежащей на $S^{m-1}(R)$, хотя бы одна координата отлична от нуля. Пусть для определённости $p_m > 0$. Тогда $p$ принадлежит верхней
полусфере

$$S^{m-1}_{+} (R) = \{x \in S^{m-1}(R): x_{m} > 0\}.$$

Но это не что иное, как график функции

$$f(x_1,\dots, x_m) = \sqrt{R^2 - x^{2}_1 - \dots - x^{2}_{m-1}},$$
которая задана в шаре пространства $R^{m-1}$. Но было доказано, что функция графика \hyperlink{example3.2}{является простым многообразиям}. Это функция класса $C^{\infty}$. Поэтому полусфера $S^{m-1}_{+} (R)$ простое многообразие класса $C^{\infty}$. Таким образом, для каждой точки существует своя полусфера. А значит получаем что сфера в $\R^m$ является $m$-мерным $\infty$-гладким многообразием. 
    
Докажем теперь, что она не является простым многообразием. Сфера компактна. Компактность это инвариант при гомеоморфизме. Но никакое открытое подмножество в $\R^k$ не является компактным. Значит не существует гомеоморфизма. Значит сфера не является простым многообразием

    \end{proof}

\end{example}

\begin{theorem}[Эквивалентные определения многообразия]
Пусть $\M \subset \R^m, m \in \N, r \in \N \cup \{+ \infty \}$. Пусть $p \in \M$.
Следующие условия эквивалентны: 
\begin{enumerate}
    \item $\exists V(p) \subset \R^m$~---~окрестность точки $p$, такая что $\mathcal{V}(p) \cap \M$ является простым $k$-мерным $r$-гладким многообразием 
    \item $\exists$ множество $\G \subset R^m$ — открытое, и диффеоморфизм $\Theta \in C^{r}(\G, \R^m)$
    
    Такое что 
$
\left\{ \begin{aligned} 
  p \in \Theta(\G)\\
  \Theta(\G) \cap \M &= \Theta(\G \cap \R^k)
\end{aligned} \right.
$
    \item Многообразие локально представляет собой пересечение $m - k$ множеств уровня r гладких функций. Те Существует открытая окрестность $U(p) \in \R^m$ и такие определенные в ней функции $F_{1}, \dots , F_{m-k} \in C^{r}(U(p), R)$, такие, что $\forall x \in U(p)$ выполнено:
    $$x \in \M \Longleftrightarrow F_{1}(x) = \dots = F_{m-k}(x) = 0$$
    А также векторы $\{grad F_1\Bigr|_{U(p) \cap \M}(p),\dots, \ grad F_{m-k}\Bigr|_{U(p) \cap \M}(p)\}$ линейно не зависимы
    \item Существует такая окрестность $W(p) \subset \R^m$, что пересечение $\M \cap W$ представляет собой график (в широком смысле) некоторого отображения класса $C^r$ , определённого в k-мерной области
\end{enumerate}
\end{theorem}

\begin{proof}
    Проведем доказательство в формате $(1) \Longrightarrow (2) \Longrightarrow (3) \Longrightarrow (4) \Longrightarrow (1)$

    $(1) \Longrightarrow (2)$ По определению простого многообразия, существует $\Phi$~---~гомеоморфизм:
    \[
    \begin{cases}
        \Phi \in C^{r}(U, \R^m) \\
        \Phi(t^0) = p \\
        \exists W(p): \Phi(U)= W(p) \cap \M \\
        \forall t \in U \hookrightarrow \rank\D \Phi(t) = k \\
    \end{cases}
    \]
    
    По теореме о продолжении до диффеоморфизма $\exists \Theta := \overline{\Phi}$. Существует открытое $\G(t^0) \in \R^m$: $\overline{\Phi} \in C^{1}(\G(t^0), \R^m)$. Получается что $\overline{\Phi}$~---~диффеоморфизм $\G(t^0)$ на $\overline{\Phi} (\G(t^0))$ \\

    Тогда получается $\Theta(\G \cap \R^k) = \overline{\Phi}(\G \cap \R^k) = \Phi(\G \cap \R^k)$ как пополнение

    С другой стороны при ограничении $\Theta$ на $\R^k$, получается, что $ \Theta = \overline{\Phi}$    \\


    $(2) \Longrightarrow (3)$
    Так как $\Theta \in C(V, \R^m)$~---~диффеоморфизм , $\exists \Theta^{-1} \in C^{1}(\Theta(\G), \R^{m})$

    Рассмотрим $\Theta^{-1} \circ \Theta (\G \cap \R^k) = \G \cap \R^{k}$ (Это пересечение — просто зануление последних $m-k$ координат)

    Тогда $\Theta^{-1} = (F_1, \dots, F_{k}, 0, \dots, 0 )$ (Заметим, что там стоят $m-k$ нулей именно для точек многообразия)

    Возьмем последние $m-k$ координатных функций отображения $\Theta ^ {-1}$.
    Получаем $F_i (\M \cap \Theta(\G)) \equiv 0$, так как равны нулю последние $m-k$ координат.
    Так как $\Theta^{-1}$ является $r$-гладким, то все $F_i$ тоже $r$-гладкие, а так как $\Theta^{-1}$ — диффеоморфизм, то $det J_{\Theta^{-1}} \neq 0$(обратима), получается что её строки и, следовательно, градиенты координатных функций линейно независимы. В частости $grad F_1\Bigr|_{U(p) \cap \M}(p),\dots, \ grad F_{m-k}\Bigr|_{U(p) \cap \M}(p)$

    $(3) \Longrightarrow (4)$
    Доказательство получается как следствие теоремы о неявном отображении. Множество уровня гладкой функции, градиент которой не обращается в ноль, локально является графиком функции, следовательно, гладким многообразием

    $(4) \Longrightarrow (1)$
    В примере многообразия \hyperlink{example3.2}{график отображения}, если график гладкий, то он является простым гладким многообразием
\end{proof}
