\begin{lemma}[Топологическое определение непрерывности]

Отображение называется непрерывным, если прообраз любого открытого множества открыт.

Те пусть $(X_1, d_1)$ и $(X_2, d_2)$~---~метрические пространства. Пусть $\Omega \subset X_1$ и $F$: $\Omega \mapsto X_2$. Тогда $$ F \in C(\Omega, X_2) \Leftrightarrow \forall V\text{~---~откр. в } X_2 \hookrightarrow F^{-1}(V) \text{~---~откр. в } X_1. $$
\end{lemma}
\begin{proof}
Идея: Если есть непрерывность, то прообраз любого эпсилон шара открыт. А любое открытое множество представимо в виде объединения открытых шаров. Обратно если прообраз любого открытого открыт, то и прообраз любого епсилон шара открыт. Пересечение открытых открыто, значит можно найти шар, где выполнено условие непрерывности.\\ 

Докажем более формально: $(\Longrightarrow)$ Запишем определение непрерывности:

$$\forall y^0 = F(x^0), \  \forall \epsilon > 0 \exists \delta > 0:\ F(B_{\delta}(x^0) \subset \Omega) \subset B_{\epsilon}(y^0).$$
Тогда $\forall$ открытого $V \subset X_2$, содержащего $y^0$, $\exists \delta(V) > 0$ $$F^{-1}(B_{\delta}(x^0) \subset \Omega) \subset B_{\epsilon}(y^0) \text{~---~открыто в} X_2.$$
Тогда $B_{\delta}(x^0) \subset F^{-1}(B_{\epsilon}(y^0))$ $\forall \epsilon > 0$ $\forall y^0 = F(x^0)$ $\Rightarrow$ $F^{-1}(B_{\epsilon}(y^0))$ является открытым.
Поскольку любое открытое множество в $X_2$ представимо в виде объединения шаров (возможно, несчётного их количества) и $$
F^{-1}(\bigcup_{\alpha \in A} V_{\alpha}) = \bigcup_{\alpha \in A} F^{-1}(V_{\alpha}),$$ то прообраз открытого множества является открытым множеством. \\
%2 ->1
$(\Longleftarrow)$: Если $\forall$ открытого $V$ в $X_2 \hookrightarrow$ $F^{-1}(V)$ ~---~ открыто в $X_1$ , то $\forall y^0$ $\forall \epsilon > 0$ $F^{-1}(B_{\epsilon}(y^0))$~---~ открыто $\Rightarrow$ $\forall\epsilon > 0$ $\forall y^0$ $\exists \delta > 0$: $B_{\delta(x^0)} \subset \Omega$ и $F(B_{\delta}(x^0) \subset B_{\epsilon}(y^0)$.
\end{proof}

\begin{question}
До этого мы рассматривали отображения $F \in \DIF(\Omega, \R^m)$, где $\Omega \subset \R^m$, $m \in \N$. А что если взять $F \in \DIF(\Omega, \R^m)$, где $\Omega \subset R^n$, $m\in \N$, $n \in \N$?
Может ли тогда образ быть открытым? Есть множество примеров, когда это не так, например:
$$A \in \mathcal{L}(\R^m, \R^n), n > m.$$
Линейное отображение переведёт $\R^m$ в его подпространство, размерность которого не больше $m$ $\Rightarrow$ Оно не будет открыто в $\R^n$, так как гиперплоскость не может быть открытым множеством в пространстве большей размерности.  \\

(Пояснение от техавших: открыто в пространстве большей размерности $\Rightarrow$ существует шар, лежащий в данном пространстве $\Rightarrow$ существует вписанный в этот шар куб той же размерности $\Rightarrow$ берем базис из рёбер данного куба — количество векторов в этом базисе равно размерности изначального пространство. Тогда из любого набора меньшего числа векторов нельзя будет выразить этот базис (по определению базиса). Отображаем из пространства меньшей размерности — получаем базис с меньшим числом векторов.) \\

Если же $n < m$, то об обратимых матрицах мы уже не можем говорить, но можем говорить о матрицах максимального ранга. 
\end{question}

\begin{theorem} (об открытом отображении версия 2) Пусть $n, m \in \N, m \geq n$; $F \in C^{1}(\Omega, \R^n)$; В каждой точке $x^0 \in \Omega \hookrightarrow \rank\D F(x^0) = n$. Тогда $F(\Omega)$~---~ открыто в $\R^n$. 
\end{theorem}
\begin{proof}
Пусть $x^0 \in \Omega$ фиксировано. С точностью до перестановки координат можно считать, что
\[
\begin{vmatrix}
  \dfrac{\partial F_1(x^0)}{\partial x_1}& \cdots & \dfrac{\partial F_1(x^0)}{\partial x_n}\\ \vdots & \ddots & \vdots \\ \dfrac{\partial F_n(x^0)}{\partial x_1}& \cdots & \dfrac{\partial F_n(x^0)}{\partial x_n}
\end{vmatrix} \neq 0 \quad \text{(матрица обратима)}
\]
Тогда $\exists \delta > 0$, такое что $B_{\delta}(x^0) \subset \Omega$ и 
\[
\begin{vmatrix}
  \dfrac{\partial F_1(x)}{\partial x_1}& \cdots & \dfrac{\partial F_1(x)}{\partial x_n}\\ \vdots & \ddots & \vdots \\ \dfrac{\partial F_n(x)}{\partial x_1}& \cdots & \dfrac{\partial F_n(x)}{\partial x_n}
\end{vmatrix} \neq 0\quad \forall x \in B_{\delta}(x^0). \eqno (\ast)
\] 
Рассмотрим сечение шара $B_{\delta}(x^0)$ подпространством, порождённым первыми $n$ координатами. Назовём получившийся шар $\widetilde{B}_{\delta}(\widetilde{x}^0)$. Рассмотрим сечение $F$ на это сечение
\[  
\widetilde{F} := F|_{\widetilde{B}_{\delta}(\widetilde{x}^0)}
\]
($\ast$) $\Rightarrow$ $\D \widetilde{F}(x)$ - обратима $\forall x \in \widetilde{B}_{\delta}(\widetilde{x}^0)$ $\Rightarrow$ применяя предыдущую теорему к отображению $\widetilde{F}$, получим, что   $\widetilde{F}(\widetilde{B}_{\delta}(\widetilde{x}^0))$~---~открыто в $R^n$ $\Rightarrow$ $F(B_{\delta}(x^0))$~---~содержит непустое открытое множество. Но $x^0 \in \Omega$ было выбрано произвольно. Тогда $\forall y^0 = F(x^0)$ $\exists$ непустое открытое множество $V$: $y^0 \in V(y^0) \subset F(\Omega)$ $\Rightarrow F(\Omega)$~---~открыто в $R^n$. \\

(Прим. ред) Те аналогично первой теореме мы предъявили, что для каждой точки образа, есть шар который вкладывается образ, а для каждой точки шара есть праобраз (именно для этого и рассматривали сужение)

\begin{note}
    В первой теореме достаточно $ F \in \DIF(\Omega, \R^m)$, но здесь необходимо именно $F \in C^{1}(\Omega, \R^n)$. Тк мы требуем, что определитель — непрерывная функция точки
\end{note}



\end{proof}
\begin{definition}
\hypertarget{homeomorphism_definition}{}
Будем говорить, что $F \in C(\Omega, \R^m)$, где $\Omega \subset \R^m$, является гомеоморфизмом, если $\exists F^{-1}$: $F(\Omega) \mapsto \Omega$, такое что $F^{-1} \in C(F(\Omega), \Omega)$.
\end{definition}

\begin{note}
    Есть теорема Брауэра(принцип инвариантности размерности). Если два открытых множества гомеоморфны, то у них одинаковые размерности. Теорему знать не обязательно, это просто интересный факт
\end{note}

\begin{definition}
    Пусть $U \subset \R^n, V \subset \R^m$~---~ открытые множества. Будем говорить, что $U$ и $V$ диффеоморфны, если существует диффеоморфизм $F$: $U \mapsto V$, то есть $F$~---~взаимооднозначное отображение $U$ на $V$, такое что $F \in C^1(U, V)$ и  $F^{-1} \in C^1(V, U)$. Если требовать именно непрерывную дифференцируемость, то такой диффеоморфизм называется гладким
\end{definition}
\begin{note}
    Пусть $U$ и $V$~---~открытые непустые множества, $U \in \R^m$, $V \in \R^n$. Если $U$ и $V$ диффеоморфны, то $m = n$.
\end{note}
\begin{proof}
Рассмотрим тождественные отображения на соответсвующее множество
    $$Id|_U = F^{-1} \cdot F, $$ $$Id|_V = F \cdot F^{-1}.$$
Так как $F$ и $F^{-1}$ дифференцируемы, то
\[
E_{m \times m} = \D F^{-1}(y^0) \cdot \D F(x^0),
\]
\[
E_{n \times n} = \D F(x^0) \cdot \D F^{-1}(y^0),
\]
где $y^0 = F(x^0)$.
Если $m \neq n$, то либо $n$-мерный образ погрузится в $m$-мерный образ, и станет $m$-мерным, либо $m$-мерный образ погрузится в $n$-мерный образ, и станет $n$-мерным, но оба случая невозможны $\Rightarrow$ n = m
\end{proof}

\begin{theorem}(производная обратного отображения) Пусть $F$: $\Omega \mapsto \R^m$, где $\Omega \subset \R^m$~---~непустое открытое множество. Пусть $x^0 \in \Omega$ и выполнены следующие условия:
\begin{enumerate}
    \item $y^0 = F(x^0)$~---~внутренняя точка для $F(\Omega)$
    \item Пусть $\exists F^{-1}$: $F(\Omega) \mapsto \Omega$
    \item Пусть $F^{-1}$ непрерывно в точке $y^0$
    \item Отображение $F$ дифференцируемо в точке $x^0$ и $\det \D F(x^0) \neq 0$ (матрица Якоби обратима)
\end{enumerate}
Тогда $F^{-1}$ дифференцируемо в точке $y^0$, и, кроме того,
\[
\D F^{-1}(y^0) = (\D F(x^0))^{-1}.
\]
\end{theorem}
\begin{proof}
    Так как $F$ дифференцируемо в точке $x^0$,
\[
F(x) - F(x^0) = \D F(x^0)(x - x^0) + \epsilon_{x^0}(x)\|x - x^0\|,\quad x \rightarrow x^0. \eqno(\ast)
\]
Так как $y^0$~---~внутренняя точка $F(\Omega)$, $\exists \epsilon > 0$ такой, что $B_{\epsilon}(y^0) \subset F(\Omega) \Rightarrow \forall y \in B_{\epsilon}(y^0)$, где $y = F(x)$, подставим в ($\ast$). Тогда
\[
y - y^0 = \D F(F^{-1}(y^0))(F^{-1}(y) - F^{-1}(y^0)) + \epsilon_{F^{-1}(y^0)}(F^{-1}(y))\|F^{-1}(y) - F^{-1}(y^0)\|. \eqno(\ast\ast)
\]
Домножим обе части ($\ast\ast$) слева на обратную матрицу, предварительно вычтя из обеих сторон второе слагаемое правой части:
\[
\left[\D F(F^{-1}(y^0))\right]^{-1}(y - y^0) - \left[\D F(F^{-1}(y^0))\right]^{-1}(y - y^0)\cdot\epsilon(F^{-1}(y)) \cdot \|F^{-1}(y) - F^{-1}(y^0)\| = 
\]
\[
= F^{-1}(y) - F^{-1}(y^0).
\]

В силу \hyperlink{lecture_6_geom_theorem}{геометрической теоремы}, $\exists \delta > 0$ $\exists c > 0$ такие, что $\overline{B_{\delta}(x^0)} \subset \Omega$ и $\|F(x) - F(x^0)\| \geq c\|x - x^0\|$ $\forall x \in \overline{B_{\delta}(x^0)}$. В силу непрерывности $F^{-1}$ в точке $y^0$ $\exists \epsilon > 0$ такой, что $\forall y \in B_{\epsilon}(y^0) \hookrightarrow F^{-1}(y) = x \in B_{\delta}(x^0)$.
Теперь докажем, что второе слагаемое в выражении выше является $o(\|y - y^0\|)$:
\[
\Big{\|}-(\D F(F^{-1}(y^0)))^{-1}(y - y^0)\cdot \epsilon(F^{-1}(y)) \cdot \|F^{-1}(y) - F^{-1}(y^0)\|\Big{\|} \leq
\]
\[ \leq \|\D F(F^{-1}(y^0))^{-1}\| \cdot \|\epsilon(F^{-1}(y))\| \cdot \|F^{-1}(y) - F^{-1}(y^0)\| \leq 
\]
подставим $y = F(x) \Leftrightarrow x = F^{-1}(y)$ в геометрическую теорему, продолжая неравенство
\[
\leq \|\D F(x^0)^{-1} \| \cdot \|\epsilon(F^{-1}(y))\| \cdot \|y - y^0\| \cdot \dfrac{1}{c} = B \eqno(\ast\ast\ast)
\]
Так как $\lim_{x \rightarrow x^0} \epsilon(x) = 0$ и $F^{-1}$ непрерывно в точке $y^0$, $\epsilon(F^{-1}(y)) \rightarrow 0$, $y \rightarrow y^0$ $\Rightarrow \|\epsilon(F^{-1}(y))\| \rightarrow 0$, $y \rightarrow y^0$.
Тогда, воспользовавшись этим вместе с ($\ast\ast\ast$), получим, что $F^{-1}$ дифференцируемо в точке $y^0$, так как представимо в виде линейного оператора и о-малого при $y \rightarrow y^0$, и при этом
\[
    \D(F^{-1})(y^0) = (\D F(F^{-1}(y^0)))^{-1}.
\]
\end{proof}

\begin{theorem}
(теорема о диффеоморфизме) Пусть $F \in C^{1}(\Omega, \R^m)$, где $\Omega \subset \R^m$~---~непустое открытое множество. 
Пусть:
\begin{enumerate}
    \item $\D F(x^0)$ обратима $\forall x^0 \in \Omega$.
    \item $\exists$ обратное отображение $F^{-1}$: $F(\Omega) \mapsto \Omega$ (Мы требуем существование)
\end{enumerate}
Тогда $F$ является диффеоморфизмом $\Omega$ на $F(\Omega)$.
\end{theorem}
\begin{proof}
    В силу теоремы об открытом отображении $F(\Omega)$~---~открыто. Остаётся доказать, что $F^{-1}$~---~дифференцируемо в $\forall$ точке $y^0 \in F(\Omega)$ и $\D F^{-1}(y)$~---~непрерывна как функция точки $y$. \\
    Пусть $U \subset \Omega$~---~открытое множество. По теореме о открытом отображении, $F(U)$~---~открыто. Но тогда $V$ такое, что $U = F^{-1}(V)$, открыто в $\R^m$. Тогда, если для $F$ образом $\forall$ $U \in \Omega$ является открытое $V \in \R^m$, $U$ = $F^{-1}(V)$ (из-за взаимооднозначности $F$ следует, что $(F^{-1})^{-1}$) $\Rightarrow$ для $F^{-1}$ прообраз $\forall$ открытого открыт $\Rightarrow$ $F^{-1}$ непрерывно $\Rightarrow$ по только что доказанной теореме о производной обратного отображения
    \[
    \forall y^0 \in F(\Omega)\quad \exists \D F^{-1}(y^0) = (\D F(F^{-1}(y^0)))^{-1}.
    \]
    Остаётся доказать, что отображение $y \mapsto (\D F(F^{-1}(y)))^{-1}$ непрерывно.
    \[
    y \mapsto F^{-1}(y) \mapsto \D F(F^{-1}(y)) \mapsto F(F^{-1}(y)))^{-1}
    \]
    Первые два отображения в данной цепочке являются непрерывными. Третье является непрерывным, так как операция взятия обратной матрицы является непрерывным отображение на множестве обратимых матриц, что мы сейчас докажем.
\end{proof}

\begin{explanation} (почему операция взятия обратной матрицы является непрерывным отображением)
\begin{enumerate}
    \item Множество всех обратимых матриц $m \times m$~---~открыто, так как $\det A$ является непрерывной функцией элементов $A$. Матрицу $A$ можно отождествить с точкой в пространстве $\R^{m^2}$; взятие $\det$ является непрерывной алгебраической функцией, а значит $\exists B_{\delta}(A)$, все точки которого, задающие матрицы, являются невырожденными $\Leftrightarrow$ обратимыми.
    \item $A^{-1}$ из своего алгоритма вычисления является непрерывной функцией элементов $A$. 
\end{enumerate}
Что нам и требовалось доказать.
\end{explanation}



% Эта теорема написана в следующей лекции, это всё по идее можно удалить
% \begin{theorem} (теорема о локальной обратимости отображения) Пусть $F \in C^{1}(\Omega, \R^m)$, где $\Omega \subset \R^m$~---~непустое открытое множество. Если $\D F(x^0) \neq 0$, то $\exists$ открытое $U \in \Omega$ и $\exists$ открытое $V \in F(\Omega)$ такие, что $F$ осуществляет диффеоморфизм $U$ на $V$.
%\end{theorem}
%\begin{proof}
 %   Так как $\det \D F(x^0) \neq 0$, по теореме о диффеоморфизме $\exists \delta > 0$ $\exists c > 0$ такой, что $\|\D F(x^0)(x - x^0)\| \geq c \| x - x^0\|$ $\forall x \in \overline{B_{\delta}(x^0)} \subset \Omega$. Но $\det \D F(x)$ является непрерывной функцией от $x$ $\Rightarrow$ $\exists \widetilde{\delta} > 0$ такой, что $\|\D F(x) - \D F(x^0)\| < \dfrac{c}{2}$ $\forall x \in B_{\widetilde{\delta}}(x^0)$
%\end{proof}