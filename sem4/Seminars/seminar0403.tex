\newpage

\section{Семинар 5}

\begin{problem}
    Вычислить $\sum\limits_{k = 1}^{\infty} \dfrac{1}{k^4}$
\end{problem}

\begin{answer}
    Решим задачу при помощи равенства Парсеваля. Напомним, как оно выглядит:

    $$
    ||f||^2 = \int\limits_{-\pi}^{\pi} |f(x)|^2 dx = 2\pi |a_0(f)|^2 + \sum\limits_{k = 1}^{\infty} \pi a_k^2(f) + \pi b_k^2(f)
    $$

    \noindent Рассмотрим функцию $f(x) = x^2$ на $[-\pi, \pi]$. Продолжим её периодически и разложим её в ряд Фурье. Так как $f$ ~---~ четная, то $b_k(f) = 0$.

    \noindent По определению:

    $$
    a_0(f) = \dfrac{1}{2\pi} \int\limits_{-\pi}^{\pi} x^2 dx = \dfrac{\pi^2}{3}
    $$

    $$
    a_k(f) = \dfrac{1}{\pi} \int\limits_{-\pi}^{\pi} x^2 \cos (kx) dx = \dfrac{4\pi (-1)^k}{k^2}
    $$

    \noindent Подставим в равенство Парсеваля:
    $$
    \int\limits_{-\pi}^{\pi} x^4 dx = ||f||^2_{L_2} = 2\pi |a_0(f)|^2 + \sum\limits_{k = 1}^{\infty} \dfrac{16\pi}{k^4} \Rightarrow
    $$
    $$
    \sum\limits_{k = 1}^{\infty} \dfrac{1}{k^4} = \dfrac{\pi^4}{90}
    $$
\end{answer}

\begin{problem}
    Вычислить $\sum\limits_{k = 1}^{\infty} \dfrac{1}{k^6}$
\end{problem}

\begin{answer}
    Нужно проделать аналогичные рассуждения для $f(x) = x^3$. Несмотря на то, что $f$ ~---~ нечетная и будут разрывы (поточечной сходимости не будет), мы переодически продолжим $f$ и сходимость в смысле $L_2$ будет $\Rightarrow$ всё хорошо.
\end{answer}

\begin{problem}
    Как связаны условия:

    \begin{enumerate}
        \item $\{e_k\}_{k = 1}^{\infty}$ в Л.Н.П. $E$ есть полная система
        \item $\{e_k\}_{k = 1}^{\infty}$ в $E$ ~---~ топологический базис.
    \end{enumerate}
\end{problem}

\begin{answer}
    Из 1 не следует 2. Пример :  тригонометрическая система в пространстве $C^*([-\pi, \pi])$ ~---~ линейное пространство непрерывных функций, у которых значения на концах равны с нормой как в $C([-\pi, \pi])$. Не является базисом, так как есть пример Шварца.
\end{answer}

\begin{example}[Неполное евклидово пространство и замкнутая, но неполная система]

\noindent Рассмотрим пространство  $\ell_2 := \{x = (x_1, \dots, x_n, \dots) : \sum\limits_{k = 1}^{\infty} x_k^2 < +
\infty\}$.

\noindent $||x||_{\ell} = \left(\sum\limits_{k = 1}^{\infty} x_k^2\right)^{\frac{1}{2}}$ ~---~ полное.

\noindent Рассмотрим базис в $\ell_2$ : $\{e_k\}_{k = 1}^{\infty}, \ e_k = (0, \dots, 0, 1, \dots)$ ~---~ единица на $k$-ом месте.

\noindent Рассмотрим вектор $e = (1, \frac{1}{2}, \frac{1}{3}, \dots, \frac{1}{n}, \dots)$ и пространство $E = Lin\{e, e_2, e_3, \dots, e_n, \dots \}$ ~---~ все конечные линейные комбинации. $E$ неполное, так как вектор $e_1 = (1, 0, \dots)$ не может быть выражен никакими линейными комбинациями.

$$
\left|\left| e_1 - (e - \sum\limits_{k = 1}^{n} \dfrac{e_k}{k}) \right|\right| = \left(\sum\limits_{k = n + 1}^{\infty} \dfrac{1}{k^2} \right)^{\frac{1}{2}} \rightarrow 0, n \rightarrow +\infty \Rightarrow
$$

$$
\left \{ e - \sum\limits_{k = 1}^{n}\right\}_{k = 1}^{\infty} \dfrac{e_k}{k}\text{~---~ фундаментальная в } E \text{, но не имеет предела в } E.
$$

\noindent Действительно, если $f \in E$  и $f \perp e_k \ \forall k \ge 2$. Но тогда у $f$ ненулевая проекция только на $e_2$. Но тогда $f = c \cdot e_2 \notin E$. То есть, система замкнута, но неполна.
\end{example}

\begin{problem}
    Является ли тригонометрическая система полной в пространстве $C([-\pi, \pi])$?
\end{problem}

\begin{answer}
    Нет, так как $\max\limits_{x \in [-\pi, \pi]} \left|x - \sum\limits_{k = 1}^{n} c_k \cdot \cos(kx) + b_k \sin (kx) + c_0 \right| \ge \pi$
\end{answer}

\begin{problem}
    Полна ли система $\{x^{2k - 1}\}_{k = 1}^{\infty}$ в $C([1, 2])$?
\end{problem}

\begin{answer}
    Продолжим функцию на $[0, \pi]$ линейно так, чтобы $f(0) = f(\pi) = 0$, а на $[-\pi, \pi]$ продолжим нечетным образом и обозначим продолжение как $F$. Далее продолжим периодически.

    \noindent По теореме Фейера $F$  приближается тригонометрическими полиномами, то есть $\forall \epsilon > 0 \ \exists N(\epsilon) : \ \forall n \ge N(\epsilon) \hookrightarrow \left|\left| F - \Sigma_n(f)\right|\right| < \frac{\epsilon}{2} \ \forall n \ge N(\epsilon).$ Здесь $\Sigma_n(f) = \frac{1}{n} \sum\limits_{k = 0}^{n - 1}S_k(f)$ ~---~ сумма Фурье.

    \noindent При этом, так как мы продолжили нечетным образом $F$, то $\Sigma_n(f) = \sum\limits_{k = 1}^{n} c_k \cdot \sin(kx)$ так как $\sin(kx)$ раскладывается в ряд Тейлора $\forall n \in \N$. Значит каждый $\sin x, \dots \sin (nx)$ можно так точно приблизить алгебраическими полиномами нечетной степени, что будет существовать полином $P_n(x)$ из нечетных степеней такой, что $\sup\limits_{x \in [-\pi, \pi]} \left|P_n(x) - \sum\limits_{k = 1}^{n} c_n\sin (kx)\right| < \frac{\epsilon}{2}$. Следовательно, система полна.

\end{answer}